\section {Research Methodology}

\subsection {Categorization Criteria}
This study aims to systematically classify the state-of-the-art in the field of Natural Language Processing combined with Software Testing 
processes. Through this procedure, this paper also targets to track possible trends and directions in current practices and techniques. In that 
way, possible future research opportunities are being also detected. To help with the categorization of the approaches and in order to give to this review a 
clear structure, we identify several categorization criteria based on which the present classification is performed. These criteria came up during the study of 
the different proposed applications. 

\subsubsection* {Criterion 1: Contribution Type}
This criterion refers to the kind of the approach that is being proposed. There is a number of possible types identified. Some of them include a new tool, 
a whole framework, a simple algorithm performing a small testing-related task (or not) or some of them may even contain a complex neural network. This paper embraces that kind of 
diversity since in that way, the different possible applications of NLP can be given prominence. We study the contribution type criterion since during the paper analysis we 
noticed the variety of existing approaches that are being proposed and developed by the scientific community. Thus, we would like to give an answer to the question regarding 
the number of these approach types, their complexity and, eventually, the way they achieve their target goal.

\subsubsection* {Criterion 2: Software Testing Stage}
A significant criterion based on which we can classify the papers studied is the stage of Software Testing that they refer to. Does NLP have many applications in the Testing field? 
Maybe there are some approaches focusing on less popular testing phases? We include the Software Testing Stage criterion in this review, as an attempt to answer questions similar 
to the ones stated above. Every approach is possible to focus on a different testing phase. For example, some of them might refer to Requirement Analysis, Test Case Design and Development, Test 
Execution etc. Through this criterion, we are able to take a look into the broader spectrum of NLP applications in Software Testing. Moreover, this is a great chance to take a 
look into other testing phases apart from test case creation, which seems to be the main topic of interest when it comes to NLP and Testing.

\subsubsection* {Criterion 3: Software Testing Type}
It is widely known that there is a significant number of different tests which are being executed to check a different aspect of an application. Does NLP have anything to do with all these types? Is it possible to 
contribute in any way on many of them? Maybe NLP is not always the answer? These kind of questions were generated during our analysis and they are also the motivation behind the Testing Type criterion of this study. 
The applications studied in this paper each refer to some type of Software Testing. Some examples are Manual Testing, Automation Testing and Security Testing. 
In this section, the broader contribution of NLP in Software Testing is projected, while the reader can also identify possible usages of NLP on different aspects of testing.

\subsubsection* {Criterion 4: Input Type}
There is also much interest about the different inputs of the proposed approaches. Reading through the studies, we notice a variety of data  
being processed to achieve the targeted result. Our questions regarding the possible contribution of NLP to different testing types lead us to further research about the information which practically builds 
the proposed approaches. What does this method need in order to perform this task? Should we provide it with complex data or its not that complicated after all? Given those questions, we want to map the level of 
complexity of the provided information on the papers' approaches. Several applications may accept as input a UML 
diagram, a graph/network or text written in a Domain Specific Language (DSL). These are some of the different types spotted during the analysis of the papers.

\subsubsection* {Criterion 5: Output Type}
Another characteristic which is closely related with the one we just mentioned is the output type of the approaches. Since we study suggested applications 
from different stages of the Software Testing lifecycle, the outputs can vary. Depending on the stage they refer to, they can have different characteristics. Moreover, depending on the task they perform and its 
level of complexity, the output type will be different. The output is also a strong indicator of the overall goal of the approach. For example, if a tool generates test case code, then possibly its target is to 
automate the test case creation process to a certain degree. However, this can be determined by taking into consideration the input type as well. Apart from the most common output type we will come across here, which is the 
test case code, another interesting aspect in this section is that some approaches may even create something which may act as input to another program, increasing that way the level of flexibility in testing.

\subsubsection* {Criterion 6: NLP Techniques}
In that case, we are interested in the type of NLP technique that is being used to perform the proposed approach. Which are those NLP methods frequently used to improve testing? What NLP process is usually 
followed on those approaches? What types of analyses are being performed? For example, some applications may use the very well known methods 
like the TF-IDF statistic, whereas others may use something less popular or even create a completely new NLP method. By categorizing the approaches 
according to this criterion, we get a better understanding of the latest NLP method trends that are being utilized and also, we identify new needs that have emerged in the 
NLP area by studying any new proposed NLP methods. This section contains significant information and is also one of the major focus knowledge areas of the present review.

\subsection {Search and Selection of papers}
The search process of the papers started by defining the corresponding search engines. The sources used were the ACM Digital Library\footnote{https://dl.acm.org/}, Scopus\footnote{https://www.scopus.com/standard/marketing.uri}, 
IEEE Xplore\footnote{https://ieeexplore.ieee.org/Xplore/home.jsp} and Google Scholar. Then, we define the search terms and keywords. The ones used were \emph{(nlp AND software testing), (nlp 
AND software testing) OR (nlp AND test automation)} and \emph{(nlp AND test automation)}. Due to the fact that the search results corresponded 
to thousands of papers, we excluded the papers that were written before the year 2000 to restrict this number. A big number of the papers that came up 
were not related at all to the current research area of interest. Also, we performed targeted search on several Conferences and Scientific Journals such 
as the International Conference on Software Engineering and the International Conference on Automated Software Engineering. The references of other previously 
published reviews were also another source utilized to track possible related studies. \\

After accessing the search results, the next step was to exclude a significant number of them and keep the ones that somehow correspond to the field we study. 
To achieve that, we read the titles and abstracts of those papers and we ended up with a total of 86 papers. Then, to further comprehend the contents of those 86 studies, 
we read and analyzed the full proposed approach of each one of them. During the reading, we excluded papers that generally refer to the field of 
Machine Learning without focusing on NLP. Also, papers that studied the broader area of Software Engineering and not specifically Testing were not 
of our interest. Many other papers studied the need for Automation in Software Testing making a short reference to the possible contribution of Machine 
Learning to that. These papers were also excluded during the searching process. \\

At that stage, all the papers of interest had been gathered and analyzed. Also, the ones that were not related to the goal of this study had been 
eliminated. After completing the above procedure, we get a final total consisting of 44 papers. 