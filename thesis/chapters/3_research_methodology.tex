\chapter {Research Methodology}

\section {Research Questions}
This study aims to systematically classify the state-of-the-art in the field of Natural Language Processing combined with Software Testing 
processes. Through this procedure, this paper also targets to track possible trends and directions in current practices and techniques. In that 
way, possible future research opportunities are being also detected. To help with the categorization of the approaches and in order to give to this review a 
clear structure, we raise several research questions (RQ). These questions came up during the study of the different proposed applications. 

\subsection {RQ1: What is the proposed contribution type?}
This question refers to the kind of the approach that is being proposed. There is a big number of possible types identified. Some of them include a new tool, 
a whole framework, a simple algorithm performing a small testing-related task (or not) or it can even be a complex neural network. This paper embraces that kind of 
diversity since in that way, the different possible applications of NLP can be given prominence.

\subsection {RQ2: To what stage of Software Testing does the approach refer to?}
A significant criterion based on which we can classify the papers studied is the stage of Software Testing that they refer to. Every approach is possible to focus 
on a different testing phase. For example, some of them might refer to Requirement Analysis, Test Case Design and Development, Test Execution etc. Through this Research 
Question, we are able to observe the broader spectrum of NLP in Software Testing. Moreover, this is a great chance to take a look into other testing phases apart from 
test case creation, which seems to be the main topic of interest when it comes to NLP and Testing.

\subsection {RQ3: In which type of Software Testing is the approach being used?}
The applications studied in this paper each refer to some type of Software Testing. Some examples of these types are Manual Testing, Automation Testing and Security Testing. 
In this section, the broader contribution of NLP in Software Testing is projected, while the reader can also identify possible usages of NLP on different aspects of testing.

\subsection {RQ4: What is the input of the approach?}
There is also much interest about the different inputs of the proposed approaches. Reading through the studies, we notice a variety of data which are 
being processed to achieve the targeted result. Several applications may accept as input a UML diagram, a graph/network or text written in a Domain Specific 
Language (DSL). These are some of the different types spotted during the analysis of the papers.

\subsection {RQ5: What is the output of the approach?}
Another Research Question which is closely related with the one we just mentioned is the output type of the approaches. Since we study suggested applications 
from different stages of the Software Testing lifecycle, the outputs can vary. Depending on the stage they refer to, they can have different characteristics. 
Apart from the most common output type we will come across here, which is the test case code, some approaches may even create something which may act as 
input to another program.

\subsection {RQ6: What NLP Technique is being used?}
In that case, we are interested in the type of NLP technique that is being used to perform the proposed approach. Some applications may use the very well-known methods 
like the TF-IDF statistic and the Bag-Of-Words Model, whereas others may use something less popular or even create a completely new NLP method. By categorizing the approaches 
according to this criterion, we get a better understanding of the latest NLP method trends that are being utilized and also, we identify new needs that have emerged in the 
NLP area by studying any new proposed NLP methods.

\section {Search and Selection of papers}
The search process of the papers started by defining the corresponding search engines. The sources used were the ACM Digital Library, Scopus, 
IEEE Xplore and Google Scholar. Then, we define the search terms and keywords. The ones used were \emph{(nlp AND software testing), (nlp 
AND software testing) OR (nlp AND test automation)} and \emph{(nlp AND test automation)}. Due to the fact that the search results corresponded 
to thousands of papers, we excluded the papers that were written before the year 2000 to restrict this number. A big number of the papers that came up 
were not related at all to the current research area of interest. Also, we performed targeted search on several Conferences and Scientific Journals such 
as the International Conference on Software Engineering and the International Conference on Automated Software Engineering. The references of other previously 
published reviews were also another source utilized to track possible related studies. \\

After accessing the search results, the next step was to exclude a significant number of them and keep the ones that somehow correspond to the field we study. 
To achieve that, we read the titles and abstracts of those papers and we ended up with a total of 82 papers. Then, to further comprehend the contents of those 82 studies, 
we read and analyzed the full proposed approach of each one of them. During the reading, we excluded papers that generally refer to the field of 
Machine Learning without focusing on NLP. Also, papers that studied the broader area of Software Engineering and not specifically Testing were not 
of our interest. Many other papers studied the need for Automation in Software Testing making a short reference to the possible contribution of Machine 
Learning to that. These papers were also excluded during the searching process. \\

At that stage, all the papers of interest have been gathered and analyzed. Also, the ones that were not related to the goal of this study have been 
eliminated. After completing the above procedure, we get a final total consisting of 40 papers. 