\chapter{Conclusion and Future Work}

This thesis paper classified the state-of-the-art and the practices of NLP used in the Software Testing process. We reviewed 44 scientific papers written after the year 2000 that focus on this 
knowledge area and contribute to different testing tasks. This Systematic Literature Review organizes the already existing knowledge of the scientific literature and it is a well structured source of 
information for both researchers and practitioners who operate in the software testing field.\\

Also, the incorporation of a popular Machine Learning process --- NLP --- in this knowledge area increases the level of alignment of this review with current practice and trends. By creating this study, 
we aim to make current practices and upcoming needs accessible to anyone interested. We simplified the searching process by providing recent organized information which can be used as a starting point 
for further research.\\

Given that, the present review motivates the business community to utilize outcomes and results of this paper in order to express or verify current needs and challenges faced in technical processes during 
practice. The scientific community is encouraged to use this study and the above expressed needs and challenges as a source of information in order to contribute to the solution of those problems.
