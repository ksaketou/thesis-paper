\chapter*{Backround and Related Work}

\section*{Backround}
text

\section*{Related Work}

Several other papers have been created that are close to the currently studied area. Four such studies where found \cite{garousi2020nlp,battina2019artificial,ahsan2017comprehensive,escalona2011overview} 
and are shortly presented in this section. Garousi et al. \cite{garousi2020nlp}, conducted a Systematic Literature Mapping regarding NLP-assisted software testing methods suggested in a pool 
of 67 papers. This mapping study classifies and overviews a list of 38 identified tools. Goal of this paper was to provide practicioners with useful information about the reasons for which 
the use of such tools might benefit their current practise. Moreover, Ahsan et al \cite{ahsan2017comprehensive}, investigated 16 papers which contained a total of 6 NLP-based techniques and 
18 tools with the same purpose. \\

Trudova et al \cite{battina2019artificial}, created a Systematic Literature Review aiming to highlight the role of Artificial Intelligence in Automation Testing. 
They explore the influence of AI on the testing processes without focusing only on NLP techniques. The authors reviewed 34 papers where they identified 9 distinct software testing activities. 
The last study identified was the one conducted by Escalona et al \cite{escalona2011overview}, which focuses on NLP techniques regarding the automation of the test cases generation process. 
The analysis was performed between 24 approaches with a goal to determine the state of the art in the related area. The summary of the identified studies is displayed in Table 1 below.\\

\begin{table}
\resizebox{\columnwidth}{!}{%
\begin{tabular}{ |p{3cm}||p{3cm}|p{3cm}|p{3cm}|p{3cm}|p{3cm}|  }
    \hline
    \hline
    \multicolumn{6}{|c|}{Table 1. SLRs identified about NLP and Software Testing} \\
    \hline
    Paper Title& Year & Reference & Papers studied & Focus Area & Empasis on\\
    \hline
        NLP-assisted software testing: A systematic mapping of the literature   & 2020    & \cite{garousi2020nlp}&   67 & NLP & NLP Tools \\
        \hline Artificial Intelligence in Software Test Automation: A Systematic Literature Review & 2020 & \cite{battina2019artificial} &  34 & Artificial Intelligence & Software Testing field\\
        \hline A Comprehensive Investigation of Natural Language Processing Techniques and Tools to Generate Automated Test Cases &   2017  & \cite{ahsan2017comprehensive}& 16  & NLP & Test Case Generation\\
        \hline An overview on test generation from functional requirements   & 2011 & \cite{escalona2011overview} &  24 & NLP & Test Case Generation\\
        \hline This Paper & 2022 &  & ??? & NLP & Software Testing field\\
    \hline
\end{tabular}%
}
\end{table}



Having made a short reference to the already existing work, we notice that none of the above papers focus on NLP techniques, different application types (tool, framework, algorithm etc.) and 
different stages of software testing at the same time. In this paper, we emphasize on studies that refer to NLP techniques, possibly propose a variety of approach types and can improve different phases 
of the Software Testing stages (e.g. requirement analysis, implementation, execution). We aim to form a broader overview of the way that NLP and Software Testing fields now cooperate to optimize 
the quality of software.