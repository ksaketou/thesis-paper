\chapter{Background and Related Work}

This section sets the scene around the topic areas that will be discussed in this paper. Firstly, we describe basic information about the concept of NLP and we also 
provide some context related to Software Testing phases and processes. Then we briefly present studies we found that are similar to this one, compare them and consequently, 
identify the gap that this paper targets to fill. 

\section {Overview of Software Testing and NLP}

Some kind of software is now present in every form of computer that exists these days, whereas no daily activity can be completed without using one. However, this leads to the expectations we 
form for that software to be extremely high since it has to perform at its best every time it is being used. Consequently, this has created the need to ensure the overall quality and functionality of that 
software. That is the role of Software Testing. According to Bourque et al \cite{swebok}, ``\emph{Software testing consists of the dynamic verification that a program provides expected behaviors on a 
finite set of test cases, suitably selected from the usually infinite execution domain}''.\\

The testing process has four key issues \cite{swebok}. The first one is that these days it is necessary to be performed using dynamic inputs and not static ones in order to ensure the adaptability 
of the program under test. Secondly, the number of test cases created should not be infinite. Even the simplest program can lead to 
countless possible test cases, which are obviously impossible to create and execute due to limitations in time and resources. The testing process gets even more chaotic when we refer to the testing of complex 
applications. Because of that, when it comes to testing it is important to create the suitable amount of tests which is adequate to assure the quality of the program under test. The third issue is that the 
selection of the testing techniques that will be used, should be performed very carefully, taking into consideration different aspects of the program to be tested, such as its functionality and other risks 
involved. Finally, the testing outcomes should be clear and easy for the test engineer to estimate if they are acceptable. In any other case, it means that the tests executed do not provide any 
substantial information for the condition and the quality of the software. \\

Another significant knowledge area that this paper focuses on is Natural Language Processing (NLP). According to Liddy E.D. \cite{liddy2001natural}, ``\emph{Natural Language Processing is a theoretically motivated 
range of computational techniques for analyzing and representing naturally occurring texts at one or more levels of linguistic analysis for the purpose of achieving human-like language processing for a 
range of tasks or applications}". NLP aims to process text with a human-like approach, by transforming textual information into a structure understandable by the computer. Nowadays, NLP is an intensely growing 
field and it keeps being constantly developed. It can be applied in a variety of ways, but Information Extraction (IE) is one which we will frequently come across in this paper. \\

Information Extraction aims to identify information structures and relationships between entities included in these structures from unstructured sources \cite{infoextraction}. It is now a very well known 
method in the NLP community with many applications in fields like machine learning, databases, web, document analysis and others. Sarawagi \cite{infoextraction} refers to these applications by separating them 
into several categories with some of them being Enterprise Applications, Personal Information Management, Scientific Applications and Web Oriented Applications. Many of the ones explained in these categories 
are related to document processing (e.g. organization of files and projects), databases (e.g. data cleaning, management of citation databases) and management of user activity on the web (e.g. comparison shopping 
websites, placement of ads in webpages etc.). The wide usage of IE is, thus, very clear based on the different types of applications referred above and it is no surprise that many of the approaches studied in 
this paper utilize IE. \\

At the below section, we present the results of the research that was made in order to map the already existing Literature Reviews related to the subject we study in this paper.

\section{Related Work}

Several other papers have been created that are close to the current research area. Four such studies where found \cite{garousi2020nlp,battina2019artificial,ahsan2017comprehensive,escalona2011overview} 
and are shortly presented in this section. Garousi et al. \cite{garousi2020nlp}, conducted a Systematic Literature Mapping regarding NLP-assisted software testing methods suggested in a pool 
of 67 papers. This mapping study classifies and overviews a list of 38 identified tools. The goal of this paper was to provide practitioners with useful information about the reasons for which 
the use of such tools might benefit their current practice. However, it seems that on the spotlight of the study exist the different NLP tools proposed in the papers. The authors emphasize on 
these tools by further reviewing them, whereas other techniques or contribution types that may be included in the papers studied don't seem to take significant part of this Systematic Literature 
Review. Moreover, Ahsan et al \cite{ahsan2017comprehensive}, investigated 16 papers which contained a total of 6 NLP-based techniques and 18 tools with the same purpose. In addition to that, this 
study focuses on papers that contribute to the field of test case generation.\\

Trudova et al \cite{battina2019artificial} created a Systematic Literature Review aiming to highlight the role of Artificial Intelligence in Automation Testing. 
They explore the influence of AI on the testing processes without focusing only on NLP techniques. The authors reviewed 34 papers where they identified 9 distinct software testing activities. 
The last study identified was the one conducted by Escalona et al \cite{escalona2011overview}, which focuses on NLP techniques regarding the automation of the test cases generation process. 
The analysis was performed between 24 approaches with a goal to determine the state of the art in the related area. The summary of the identified studies is displayed in \hyperref[table1]{Table 1}. \\

\begin{table}
\resizebox{\columnwidth}{!}{%
\begin{tabular}{ |p{3cm}||p{3cm}|p{3cm}|p{3cm}|p{3cm}|p{3cm}|  }
    \hline
    \hline
    Title & Year & Reference & Papers studied & Focus Area & Empasis on\\
    \hline
        NLP-assisted software testing: A systematic mapping of the literature   & 2020    & \cite{garousi2020nlp}&   67 & NLP & NLP Tools \\
        \hline Artificial Intelligence in Software Test Automation: A Systematic Literature Review & 2020 & \cite{battina2019artificial} &  34 & Artificial Intelligence & Software Testing field\\
        \hline A Comprehensive Investigation of Natural Language Processing Techniques and Tools to Generate Automated Test Cases &   2017  & \cite{ahsan2017comprehensive}& 16  & NLP & Test Case Generation\\
        \hline An overview on test generation from functional requirements   & 2011 & \cite{escalona2011overview} &  24 & NLP & Test Case Generation\\
        \hline This Paper & 2022 &  & 44 & NLP & Software Testing field\\
    \hline
\end{tabular}%
}
\caption{SLRs identified about NLP and Software Testing}
\label{table1}
\end{table}

We noticed that the wider knowledge area of Software Engineering is frequently studied as well, along with Natural Language Processing. We detected several papers related 
to different stages of that area and we state some of them in the following section. The first paper studies NLP during the Requirements Engineering (RE) process. 
RE contains activities related to maintaining, managing and creating the requirements in the design process. It is an essential step for the proper developing of 
every system and application in order to align with the corresponding specifications. Dalpiaz \& Brinkkemper \cite{reqeng1} propose a whole pipeline 
which combines several tools previously proposed in the literature in order to eliminate textual defects in user stories and, thus, to make the developing 
of the product architecture an easier process. \\

Another study touches the field of Release Planning which is tightly depended on the whole progress of the 
program development. Sharma \& Kumar \cite{8701252} focus on automatically managing user stories for Release Planning during Agile development. Specifically, 
they use NLP algorithms like RV Coefficient to categorize the user stories into different releases.\\

Having made a short reference to the already existing work, we notice that none of the above papers focuses on all three of NLP techniques, different application types (tool, framework, algorithm etc.) and 
different stages of software testing. In this paper, we emphasize on studies that refer to NLP techniques, possibly propose a variety of approach types and can improve different phases 
of the Software Testing stages (e.g. requirement analysis, implementation, execution). We aim to form a broader overview of the way that NLP and Software Testing fields now cooperate to optimize 
the quality of software.