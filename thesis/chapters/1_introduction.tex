\chapter*{Introduction}

Software Testing is considered a very important stage of the Software Development Lifecycle. It aims to verify and review the quality of the whole system 
under development and to make sure that it functions according to all the requirements specified by the stakeholders. There are many different types of software testing and each 
one of them has different objectives regarding the aspect of the system that will be tested. Significant aspects include the system functionality 
(Functional/Regression/Unit Testing), its overall performance (Performance/Stress Testing), its usability and also the approval from the end user and 
the degree in which he is satisfied with the system created (Acceptance Testing). Through the years, the mindset around software testing 
has developed. It is not being faced as a procedure that happens after the development stage but, on the contrary, it is now a necessary 
step towards the successful building of software. Testing happens during the whole development lifecycle, in parallel with everything else \cite{swebok}.  \\

Because of the whole growth of the software testing mindset, it has now become a process that requires many resources in order to be sucessfully
 completed and it is now one of the most expensive procedures \cite{testautomation}. For that reason, the need for integration of Automation Testing in software testing
  processes has been constantly increasing during the last years. Thanks to the use of automated test cases and automated testing in general, many benefits can be achieved, such 
  as reduction of test costs and effective reuse of code. According to a case study performed at a law-management software \cite{introautotesting}, the creation and
   use of automated GUI test cases improved significantly the whole testing process in many ways. For example, the test execution time reduced from 2 days to 1 hour and the 
   customer satisfaction increased. \\

In addition to Software Test Automation, Machine Learning methods are being also used in order to optimize automation and make the whole testing process smarter and more 
effective. More specifically, Natural Language Processing (NLP) techniques have gained significant importance with applications that focus on different stages of the software testing process. 
A plethora of approaches have been proposed, with each one of them suggesting a different method regarding, for example, the requirement analysis phase, the test case implementation or execution. 
Some of these approaches target exclusively the NLP domain whereas others incorporate it into a wider method proposal for automation testing. The kind of the proposal may also vary from a simple 
algorithm to a whole new framework. \\

This paper is a Systematic Literature Review (SLR) study about the proposed applications of NLP methods that have been published in the last two decades and refer to the Software Testing domain. 
The already existing knowledge is being organized and presented here, filtered and structured according to the different criteria identified during the scientific literature research. Through this paper, 
a current and updated context is provided, regarding the fields of NLP and Software Testing, that corresponds to recent knowledge and applications which are accessible to anyone interested in the topics discussed.