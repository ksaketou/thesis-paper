\chapter*{ Μέρος Ι: Πτυχιακή Εργασία}

\newpage

\begin{titlingpage}
    \clearpage 
    \vspace*{\fill} 
    \begin{center} 
        \begin{minipage}
            {\textwidth}
                \begin{center}
                    {\LARGE\bfseries Natural Language Processing in Software Testing:\\ A Systematic Literature Review\par\vspace{2\baselineskip}}
                    Konstantina Saketou\textsuperscript{1}
                    \par\vspace{\baselineskip}
                    \textsuperscript{1} Athens University of Economics and Business\par \url{www.aueb.gr}\par \end{center}
                \begin{abstract}
                    \normalsize
                        The incorporation of Machine Learning and AI techniques to automate Software Testing processes has been a frequent phenomenon in recent years.
                        Natural Language Processing (NLP) is the component of this family on which we ephazise on in the present paper. Since the amount of knowledge in this 
                        area is constantly growing, there is a need to organize and provide a clear overview of the state-of-the-art technologies to both the scientific 
                        community and practitioners. This thesis paper is a Systematic Literature Review of the recent proposed NLP applications in Software Testing. We organize 
                        and classify the already existing knowledge included in a total of 44 scientific papers, based on the criterias of Contribution Type, Software Testing Stage, 
                        Software Testingg Type, Input Type, Output Type and also NLP Techniques used. Several significant findings of this review are: (1) the majority of the 
                        proposals refer to a new tool (2) many papers stick to popular and well-known NLP techniques like POS Tagging and TF-IDF to achieve their goal whereas 
                        few of them make use of more complex methods like LSTM RNNs. The present paper, aims to make current practices and upcoming needs accessible by simplifying 
                        the searching process and providing recent organized information that can be used as a starting point for further research.
                \end{abstract}
        \end{minipage} 
    \end{center} 
    \vfill 
    \clearpage

\end{titlingpage}


\startlist{toc}
\printlist{toc}{}{\chapter*{Contents}}
\newpage

\chapter*{Introduction}
\addcontentsline{toc}{chapter}{Introduction}

Software Testing is considered a very important stage of the Software Development Lifecycle. It aims to verify and review the quality of the whole system 
under development and to make sure that it functions according to all the requirements specified by the stakeholders. There are many different types of software testing and each 
one of them has different objectives regarding the aspect of the system that will be tested. Significant aspects include the system functionality 
(Functional/Regression/Unit Testing), its overall performance (Performance/Stress Testing), its usability and also the approval from the end user and 
the degree in which he is satisfied with the system created (Acceptance Testing). Through the years, the mindset around software testing 
has developed. It is not being faced as a procedure that happens after the development stage but, on the contrary, it is now a necessary 
step towards the successful building of software. Testing happens during the whole development lifecycle, in parallel with everything else \cite{swebok}.  \\

Because of the whole growth of the software testing mindset, it has now become a process that requires many resources in order to be sucessfully
 completed and it is now one of the most expensive procedures \cite{testautomation}. For that reason, the need for integration of Automation Testing in software testing
  processes has been constantly increasing during the last years. Thanks to the use of automated test cases and automated testing in general, many benefits can be achieved, such 
  as reduction of test costs and effective reuse of code. According to a case study performed at a law-management software \cite{introautotesting}, the creation and
   use of automated GUI test cases improved significantly the whole testing process in many ways. For example, the test execution time reduced from 2 days to 1 hour and the 
   customer satisfaction increased. \\

In addition to Software Test Automation, Machine Learning methods are being also used in order to optimize automation and make the whole testing process smarter and more 
effective. More specifically, Natural Language Processing (NLP) techniques have gained significant importance with applications that focus on different stages of the software testing process. 
A plethora of approaches have been proposed, with each one of them suggesting a different method regarding, for example, the requirement analysis phase, the test case implementation or execution. 
Some of these approaches target exclusively the NLP domain whereas others incorporate it into a wider method proposal for automation testing. The kind of the proposal may also vary from a simple 
algorithm to a whole new framework. \\

This paper is a Systematic Literature Review (SLR) study about the proposed applications of NLP methods that have been published in the last two decades and refer to the Software Testing domain. 
The already existing knowledge is being organized and presented here, filtered and structured according to the different criteria identified during the scientific literature research. Through this paper, 
a current and updated context is provided, regarding the fields of NLP and Software Testing, that corresponds to recent knowledge and applications which are accessible to anyone interested in the topics discussed.
\chapter*{Background and Related Work}
This section sets the scene around the topic areas that will be discussed in this paper. Firstly, we describe basic information about the concept of NLP and we also 
provide some context related to Software Testing phases and processes. Then we briefly present studies we found that are similar to this one, compare them and consequently, 
identify the gap that this paper targets to fill. 

\section {Overview of Software Testing and NLP}

Some kind of software is now present in every form of computer that exists these days, whereas no daily activity can be completed without using one. However, this leads to the expectations we 
form for that software to be extremely high since it has to perform at its best every time it is being used. Consequently, this has created the need to ensure the overall quality and functionality of that 
software. That is the role of Software Testing. According to Bourque et al \cite{swebok}, ``\emph{Software testing consists of the dynamic verification that a program provides expected behaviors on a 
finite set of test cases, suitably selected from the usually infinite execution domain}''.\\

The testing process has four key issues \cite{swebok}. The first one is that these days it is necessary to be performed using dynamic inputs and not static ones in order to ensure the adaptability 
of the program under test. Secondly, the number of test cases created should not be infinite. Even the simplest program can lead to 
countless possible test cases, which are obviously impossible to create and execute due to limitations in time and resources. The testing process gets even more chaotic when we refer to the testing of complex 
applications. Because of that, when it comes to testing it is important to create the suitable amount of tests which is adequate to assure the quality of the program under test. The third issue is that the 
selection of the testing techniques that will be used, should be perfomed very carefully, taking into consideration different aspects of the program to be tested, such as its functionality and other risks 
involved. Finally, the testing outcomes should be clear and easy for the test engineer to estimate if they are acceptable. In any other case, it means that the tests executed do not provide any 
substantial information for the condition and the quality of the software. \\

Another significant knowledge area that this paper focuses on is Natural Language Processing (NLP). According to Liddy E.D. \cite{liddy2001natural}, ``\emph{Natural Language Processing is a theoretically motivated 
range of computational techniques for analyzing and representing naturally occurring texts at one or more levels of linguistic analysis for the purpose of achieving human-like language processing for a 
range of tasks or applications}". NLP aims to process text with a human-like approach, by transforming textual information into a structure understandable by the computer. Nowadays, NLP is an intensely growing 
field and it keeps being constantly developed. It can be applied in a variety of ways, but Information Extraction (IE) is one which we will frequently come across in this paper. \\

Information Extraction aims to identify information structures and relationships between entities included in these structures from unstructured sources \cite{infoextraction}. It is now a very well-known 
method in the NLP community with many applications in fields like machine learning, databases, web, document analysis and others. Sarawagi \cite{infoextraction} refers to these applications by separating them 
into several categories with some of them being Enterprise Applications, Personal Information Management, Scientific Applications and Web Oriented Applications. Many of the ones explained in these categories 
are related to document processing (e.g. organization of files and projects), databases (e.g. data cleaning, management of citation databases) and management of user activity on the web (e.g. comparison shopping 
websites, placement of ads in webpages etc.). The wide usage of IE is, thus, very clear based on the different types of applications refered above and it is no surprise that many of the approaches studied in 
this paper utilize IE. \\

At the below section, we present the results of the research that was made in order to map the already existing Literature Reviews related to the subject we study in this paper.

\section{Related Work}

Several other papers have been created that are close to the current research area. Four such studies where found \cite{garousi2020nlp,battina2019artificial,ahsan2017comprehensive,escalona2011overview} 
and are shortly presented in this section. Garousi et al. \cite{garousi2020nlp}, conducted a Systematic Literature Mapping regarding NLP-assisted software testing methods suggested in a pool 
of 67 papers. This mapping study classifies and overviews a list of 38 identified tools. The goal of this paper was to provide practicioners with useful information about the reasons for which 
the use of such tools might benefit their current practice. However, it seems that on the spotlight of the study exist the different NLP tools proposed in the papers. The authors emphasize on 
these tools by further reviweing them, whereas other techniques or contribution types that may be included in the papers studied don't seem to take significant part of this Systematic Literature 
Review. Moreover, Ahsan et al \cite{ahsan2017comprehensive}, investigated 16 papers which contained a total of 6 NLP-based techniques and 18 tools with the same purpose. In addition to that, this 
study focuses on papers that contribute to the field of test case generation.\\

Trudova et al \cite{battina2019artificial} created a Systematic Literature Review aiming to highlight the role of Artificial Intelligence in Automation Testing. 
They explore the influence of AI on the testing processes without focusing only on NLP techniques. The authors reviewed 34 papers where they identified 9 distinct software testing activities. 
The last study identified was the one conducted by Escalona et al \cite{escalona2011overview}, which focuses on NLP techniques regarding the automation of the test cases generation process. 
The analysis was performed between 24 approaches with a goal to determine the state of the art in the related area. The summary of the identified studies is displayed in \hyperref[table1]{Table 1}. \\

\begin{table}
\resizebox{\columnwidth}{!}{%
\begin{tabular}{ |p{3cm}||p{3cm}|p{3cm}|p{3cm}|p{3cm}|p{3cm}|  }
    \hline
    \hline
    Title & Year & Reference & Papers studied & Focus Area & Empasis on\\
    \hline
        NLP-assisted software testing: A systematic mapping of the literature   & 2020    & \cite{garousi2020nlp}&   67 & NLP & NLP Tools \\
        \hline Artificial Intelligence in Software Test Automation: A Systematic Literature Review & 2020 & \cite{battina2019artificial} &  34 & Artificial Intelligence & Software Testing field\\
        \hline A Comprehensive Investigation of Natural Language Processing Techniques and Tools to Generate Automated Test Cases &   2017  & \cite{ahsan2017comprehensive}& 16  & NLP & Test Case Generation\\
        \hline An overview on test generation from functional requirements   & 2011 & \cite{escalona2011overview} &  24 & NLP & Test Case Generation\\
        \hline This Paper & 2022 &  & ??? & NLP & Software Testing field\\
    \hline
\end{tabular}%
}
\caption{SLRs identified about NLP and Software Testing}
\label{table1}
\end{table}

% TODO: SLR in software Engineering ??

Having made a short reference to the already existing work, we notice that none of the above papers focuses on all three of NLP techniques, different application types (tool, framework, algorithm etc.) and 
different stages of software testing. In this paper, we emphasize on studies that refer to NLP techniques, possibly propose a variety of approach types and can improve different phases 
of the Software Testing stages (e.g. requirement analysis, implementation, execution). We aim to form a broader overview of the way that NLP and Software Testing fields now cooperate to optimize 
the quality of software.
\chapter{Research Methodology}

\section {Categorization Criteria}
This study aims to systematically classify the state-of-the-art in the field of Natural Language Processing combined with Software Testing 
processes. Through this procedure, this paper also targets to track possible trends and directions in current practices and techniques. In that 
way, possible future research opportunities are being also detected. To help with the categorization of the approaches and in order to give to this review a 
clear structure, we identify several categorization criteria based on which the present classification is performed. These criteria came up during the study of 
the different proposed applications. 

\subsection {Criterion 1: Contribution Type}
This criterion refers to the kind of the approach that is being proposed. There is a number of possible types identified. Some of them include a new tool, 
a whole framework, a simple algorithm performing a small testing-related task (or not) or some of them may even contain a complex neural network. This paper embraces that kind of 
diversity since in that way, the different possible applications of NLP can be given prominence. We study the contribution type criterion since during the paper analysis we 
noticed the variety of existing approaches that are being proposed and developed by the scientific community. Thus, we would like to give an answer to the question regarding 
the number of these approach types, their complexity and, eventually, the way they achieve their target goal.

\subsection {Criterion 2: Software Testing Stage}
A significant criterion based on which we can classify the papers studied is the stage of Software Testing that they refer to. Does NLP have many applications in the Testing field? 
Maybe there are some approaches focusing on less popular testing phases? We include the Software Testing Stage criterion in this review, as an attempt to answer questions similar 
to the ones stated above. Every approach is possible to focus on a different testing phase. For example, some of them might refer to Requirement Analysis, Test Case Design and Development, Test 
Execution etc. Through this criterion, we are able to take a look into the broader spectrum of NLP applications in Software Testing. Moreover, this is a great chance to take a 
look into other testing phases apart from test case creation, which seems to be the main topic of interest when it comes to NLP and Testing.

\subsection {Criterion 3: Software Testing Type}
It is widely known that there is a significant number of different tests which are being executed to check a different aspect of an application. Does NLP have anything to do with all these types? Is it possible to 
contribute in any way on many of them? Maybe NLP is not always the answer? These kind of questions were generated during our analysis and they are also the motivation behind the Testing Type criterion of this study. 
The applications studied in this paper each refer to some type of Software Testing. Some examples are Manual Testing, Automation Testing and Security Testing. 
In this section, the broader contribution of NLP in Software Testing is projected, while the reader can also identify possible usages of NLP on different aspects of testing.

\subsection {Criterion 4: Input Type}
There is also much interest about the different inputs of the proposed approaches. Reading through the studies, we notice a variety of data  
being processed to achieve the targeted result. Our questions regarding the possible contribution of NLP to different testing types lead us to further research about the information which practically builds 
the proposed approaches. What does this method need in order to perform this task? Should we provide it with complex data or its not that complicated after all? Given those questions, we want to map the level of 
complexity of the provided information on the papers' approaches. Several applications may accept as input a UML 
diagram, a graph/network or text written in a Domain Specific Language (DSL). These are some of the different types spotted during the analysis of the papers.

\subsection {Criterion 5: Output Type}
Another characteristic which is closely related with the one we just mentioned is the output type of the approaches. Since we study suggested applications 
from different stages of the Software Testing lifecycle, the outputs can vary. Depending on the stage they refer to, they can have different characteristics. Moreover, depending on the task they perform and its 
level of complexity, the output type will be different. The output is also a strong indicator of the overall goal of the approach. For example, if a tool generates test case code, then possibly its target is to 
automate the test case creation process to a certain degree. However, this can be determined by taking into consideration the input type as well. Apart from the most common output type we will come across here, which is the 
test case code, another interesting aspect in this section is that some approaches may even create something which may act as input to another program, increasing that way the level of flexibility in testing.

\subsection {Criterion 6: NLP Techniques}
In that case, we are interested in the type of NLP technique that is being used to perform the proposed approach. Which are those NLP methods frequently used to improve testing? What NLP process is usually 
followed on those approaches? What types of analyses are being performed? For example, some applications may use the very well known methods 
like the TF-IDF statistic, whereas others may use something less popular or even create a completely new NLP method. By categorizing the approaches 
according to this criterion, we get a better understanding of the latest NLP method trends that are being utilized and also, we identify new needs that have emerged in the 
NLP area by studying any new proposed NLP methods. This section contains significant information and is also one of the major focus knowledge areas of the present review.

\section {Search and Selection of papers}
The search process of the papers started by defining the corresponding search engines. The sources used were the ACM Digital Library\footnote{https://dl.acm.org/}, Scopus\footnote{https://www.scopus.com/standard/marketing.uri}, 
IEEE Xplore\footnote{https://ieeexplore.ieee.org/Xplore/home.jsp} and Google Scholar. Then, we define the search terms and keywords. The ones used were \emph{(nlp AND software testing), (nlp 
AND software testing) OR (nlp AND test automation)} and \emph{(nlp AND test automation)}. Due to the fact that the search results corresponded 
to thousands of papers, we excluded the papers that were written before the year 2000 to restrict this number. A big number of the papers that came up 
were not related at all to the current research area of interest. Also, we performed targeted search on several Conferences and Scientific Journals such 
as the International Conference on Software Engineering and the International Conference on Automated Software Engineering. The references of other previously 
published reviews were also another source utilized to track possible related studies. \\

After accessing the search results, the next step was to exclude a significant number of them and keep the ones that somehow correspond to the field we study. 
To achieve that, we read the titles and abstracts of those papers and we ended up with a total of 86 papers. Then, to further comprehend the contents of those 86 studies, 
we read and analyzed the full proposed approach of each one of them. During the reading, we excluded papers that generally refer to the field of 
Machine Learning without focusing on NLP. Also, papers that studied the broader area of Software Engineering and not specifically Testing were not 
of our interest. Many other papers studied the need for Automation in Software Testing making a short reference to the possible contribution of Machine 
Learning to that. These papers were also excluded during the searching process. \\

At that stage, all the papers of interest had been gathered and analyzed. Also, the ones that were not related to the goal of this study had been 
eliminated. After completing the above procedure, we get a final total consisting of 44 papers. 
\chapter*{Results}
\addcontentsline{toc}{chapter}{Results}

In this section we present the results of our Systematic Literature Review and the study of the papers identified. We separately refer to 
each Research Question individually by commenting and making a short reference to the results found. In some sections and always depending on the RQ, we further analyze any 
techniques or practices encountered.

\section {Contribution Type}
After reviewing all of the papers, we ended up with several contribution types proposed in their context. At this point, we should note that some papers 
proposed more than one types so some of them are included into more than one categories. \\

The first and most encountered contribution type is that of a method technique. This type includes any proposal of approaches, processes, techniques and any 
other alternative way of performing a certain task. We grouped them all into this category since they all somehow refer to something very similar. Specifically, 20 of the
studied papers suggest such an approach of that type. 9 out those 20 papers (45\%) propose a way for generating test cases automatically from different input types, like 
use case specification and funtional requirement documents. Motwani et al \cite{8812070}, for example, present Swami which is a language-agnostic test case constructor 
based on Javascript code templates and the ECMA-262 Standard. Swami generates test case code based on the code documentation and software specifications provided. Moreover, 
Nogueira et al \cite{nogueira2015automatic} propose an event flow for test case generation from input provided in a Control Flow Language. This flow consists of several other 
processes (e.g. translation of use case descriptions) that contribute to the achievemennt of the overall goal of the proposal. \\

Another significant contribution type we distinguished is that of a tool. 14 of the studied papers suggest some sort of tool for the ease of different software testing processes. 
Some of those tools are part of the overall approach, whereas others are the main focus of the study. Some of them target the test case execution process, others focus on test case 
generation or even other areas like defects prevention and requirement analysis facilitation. Pedemonte et al \cite{pedemonte2012towards} present an approach to convert manual 
functional tests into a state ready for automatic execution by using Machine Learning methods to process textual information. This is a semi-automatic approach since in cases 
of textual information ambiguity, the user is prompted to interact with the tool to resolve the issue and provide the answer required to continue with the process. Another 
considerable approach is the CTRAS tool created by Hao et al \cite{8811987} which aims at the summarization of duplicate test reports in oprder to make them easily understandable 
by developers. The tool idenitfies similarities in both textual information and screenshots and comprehensibly presents them into a single report. \\

Several other papers focused on developing a Framework to achieve their goal. Some of them refer to the topic of test case generation while others 
emphasize on the earlier stage of requirement analysis and transformation. Lafi et al \cite{9491761} created a framework which consists of different other 
processes in order to generate test cases. First of all, the use case descriptions are being processed in order to create a control flow graph and an NLP table of the 
system under test. Based on those, test paths are then generated which they will finally lead to the test case code. As far as the requirement analysis phase 
is concerned, Viggiato et al \cite{viggiato2022using} created a framework in order to help testers improve the test cases they create. This framework 
accepts test cases in natural language as input and then provides recommendations and proposed changes. More specifically, it can provide recommendations 
for the improvement of the terminology of the test cases, it can identify potentially missing test steps from those test cases and lastly, it can 
prevent test case duplicates since it is able to identify similar test cases with the one which is currently being analyzed. All of the above three 
outputs contribute immensely to the proper and effective forming of test cases by making the testers' job much easier. \\

The last contribution type spotted is the proposal of some sort of strategy. This might be a certain way of handling or working on a specific process which 
could include the use of some specific method or tool. Yet again, the papers we identified serve different kinds of purposes on different stages 
of testing. Wong et al \cite{wong2015dase} propose DASE, a path pruning strategy which aims to improve test case execution and bug detection. It can also prevent 
defects and increase test coverage. It operates by extracting input constraints from program documentation in order to guide and form test execution paths. 
Moreover, Tahvili et al \cite{10.1145/3195538.3195540} proposed a strategy which, given test specifications in natural language, idntifies relationships 
between those test cases and finally, generates a series of proposed tets case scheduling strategies for execution. This is great way to effectively 
prioritize the test case execution process and optimize test case execution. \\

After the analysis of the identified contribution types of the papers studied, we notice that the majority of scientific contributions revovle around 
the proposal of a technique or method of performing a process. This is not something unexpected. Software Testing consists of any major processes, like 
tets case generation, test planning, defect prevention and others, which shape all together the overall purpose of this knowledge area. Thus, the fact that 
many researchers and people interested in the field have figured out ways to enhance those methods makes a lot of sense. They have achieved 
that by proposing new tehniques that be integrated to these processes and give out great results.

\section {Software Testing Stage}
Another aspect of the papers we studied that we are intersted in is the Stage of the Software Testing Lifecycle to which they apply. Some of the approaches we found refer to 
the Requirement Analysis phase, others to the Implementation and Development, Execution etc. The first one we will comment here is the Requirement Analysis. Almost 18\% of the studies 
come up with a proposal regarding that stage and they target to enhance the processing of the test requirements and specifications. This procedure happens before the construction 
of the test cases of any form and it is a critical stage which many times determines the overall outcome of the testing cycle. Thus, it is of major importance to effectivelly 
perform the Requirement Analysis for the optimal result. \\

Sainani et al \cite{reqclass}, for example, propose a method technique which idenitfies and classifies requirements from software engineering contracts. These documents contain all 
the necessary information about the product to be developed, and therefore tested, with the desired characteristics and specifications being a part of them. The developed approach 
analyzes the contract document using different NLP techniques, which we will discuss in a further section, and outputs the different requirements of the product as well as their type. 
The output might add important information regarding the product's architecture or the importance of each feature. Based on that, developers and testers get a clear overview of what 
is worth to be tested more and also the parts that need more attention during the development and testing. This method refers to very early steps on software testing. However, 
this does not mean that such a process is not significant for a successfull testing cycle. On the contrary, it sets a concrete base for all the other stages that follow. \\

Equal importance to the Requirement Analysis stage is given by Femmer et al \cite{femmer2017rapid}. They propose a tool called Smella, which identifies Requirement Smells inside 
Requirement documents. At this point, it is worth mentioning what Requirement Smells actually are. Code smells are parts of code that need to be somehow changed in order to enhance 
its structure, complexity or comprehension \cite{fowler2018refactoring}. Consequently, Requirement Smells represent spots of Requirement Documents which need to be modified in a 
certain way in order for the product's specifications to be stated with the optimal way. For example, in a certain document it is possible to have the same requirement expressed in 
a different way. This might happen due to syntax mistakes or overall use of complex syntax in the document, which makes comprehension a harder task for the reader. This is 
the job of the Smella tool. By making good use of different NLP techniques and incorporating them into the tool, the authors managed to perform the processed described above. \\

The next stage we will refer to in this section is Software Test Planning which contains tasks like Effort Estimation and the overall planning of the roadmap of the testing 
process. Yang et al \cite{9617598} propose a method called DivClass which aims to define the optimal prioritization of test reports inspection in Crowdsource Testing. This testing type 
is covered in a next section of this chapter, so we will not further explain it here. The proposed method initially performs numerous natural language processing tasks to transform the 
test report dataset it accepts as input. Then, the similarity of those test reports is calculated and finally, based on that, the inspection priority of each one is determined. This is a 
short description of the overall method. We perform detailed reference to the different NLP techniques in Section 4.6. This approach can benefit the Software Testing process since it can 
reduce the time spent on the inspection of test reports which are not that critical or that they don't contain high importance information. On the contrary, substantial amount of time 
can be spent on inspecting test reports of higher priority. Another method which targets to make Software Testing faster is the one proposed by Tahvili et al \cite{8051381}. Their approach 
predicts the execution time of manual test cases based on their textual specifications and also on historical data from previously executed test cases. Undoubtedly this approach has a lot 
to offer to the software testing community since it enhances the manual test case execution process which most of the times it ends up requiring the most amount of time to be completed. Taking 
into account that the tester now knows the approximate execution time of the test cases, it is easier for him/her to plan the overall test cycle and create the expected timetable. \\

The testing stage which gathers the majority of the attention in the scinetific literature is the implementation/development stage. That is when the test cases are created and the whole testing 
project comes to life. At this point, it is worth mentioning that almost the 70\% of the papers studied refer to that stage in some way. Because of that, we have a plethora of approach types each 
targeting a different issue. A very common task we frequently came across is the one of test case generation, usually from Requirement Documents expressed in Natural Language. Kamalakar et all 
\cite{kamalakar2013automatically}, for example, created a tool called Kirby which generates test case code from textual Requirement Specifications. Code pieces of the System Under Test 
can be also used in order for the tool to better understand the structure and semantics. However, the framework proposed by Lafi et al \cite{9491761} in order to perform the same task is different. 
This approach uses the Use Case Description of the Use Case Diagram expressed in UML. It is obvious that there is a variety of alternatives that achieve the same goals, which is the test case generation. 
Though this is not the only one we identified at the implementation stage. Viggiato et al \cite{viggiato2022using} propose a framework which aims at improving manual test case descriptions. More specifically, 
this approach accepts the test cases expressed in natural language and generates recommendations for the improvement of those descriptions. The framework recommends test case terminology improvements, 
identifies possible missing steps from those test case and it is also capable of finding potential test cases that already exist in the test code and which are similar to the one provided as input. This 
is a very powerfull approach which makes the test case implementation a piece of cake. Developing test cases based on clear, complete and comprehensive descriptions eliminates the possibility of 
mistakes and makes the whole process much faster. \\

Some of the papers we studied aim to also enhance the stage of test case execution. Pedemonte et al \cite{pedemonte2012towards} created a tool which converts manual test cases into automated ones. 
In their analysis, they emphasize on the importance of the automatic test case execution and that is why they pursuited the implementation of that task. Their approach has the ability of accepting user 
guidance if necessary, but 70\% of the test cases on which they evaluated the tool converted automatically to automated test steps without user guidance. Consequently, we see that the execution stage is also 
considered a significant one for the sucessfull completion of the Software Testing process. \\

After analyzing the testing stages on which the studied papers refer to, we should make a short reference to a less popular issue of the testing world which is that of test report manipulation. Some of the studies 
we encountered, develop approaches that aim to transform test code and outcomes into a more human-friendly format. Hao et al \cite{8811987} created a tool which not only identifies duplicate reports, but it also 
summarizes their content into a more comprehensive report. This process also happens on bug reports. On the other hand, Gonzalez et al \cite{10.1145/3283812.3283819} implemented a tool which performs a process 
opposite to what we have encountered so far in this review. The created tool converts JUnit Assertions into text written in English. Goal of this approach is to contribute to the maintenability, understandability 
and analysis of test code by reducing the existence of complex and difficult to explain code.

\chapter{Discussion}

This section provides an overview of the mapping that took place on the main part of this review based on the criteria discussed in Chapter 3. We discuss trends noticed 
in each criterion regarding the usage of some techniques or any frequent patterns identified at the studied proposed approaches. This, finally, leads us to several conclusions 
about the current trends of NLP applications in the Software Testing field.\\

Starting from the Contribution Type characteristic, we notice a big tendency of the scientific community in developing frameworks and tools. This probably happens because 
these type of methods align with today's needs, since many times they combine complex functionality with applications to trending business scenarios. Frameworks and tools are 
widely used by corporations and smaller businesses in order to function fast and effectively. Consequently, the scientific community adapts to today's practices and aims at 
improving and facilitating the current practice by developing methods which easily adapt to the business world.\\

Our analysis, then, proceeded to the Software Testing Stage criterion with the Implementation stage being on top of the identified types. This stage is both a very important and 
also a very time consuming part of the Testing lifecycle, and sometimes it may act as a barrier in the current Agile Development methodology followed by many organizations. A very 
big number of the papers we studied in this review have created approaches to automate the test case creation process, with that requiring very little human effort to be completed. 
That way, test case development becomes almost fully automated and the duration of the testing process decreases significantly. This again proves that the academic research keeps 
up with current trending practices in order to contribute new and usefull knowledge.\\

As far as the Testing Type criterion is concerned, the vast majority of the approaches focuses on Automated Testing. As we have discussed in earlier sections of this review, the test automation 
field has been rapidly growing the last few years along with the overall tendency for automation existing in numerous parts of the corporate world and also everyday life. Taking into consideration 
the need for adaptation to new changes, the authors of many of the papers we refered here have brought automation processes one step closer to business practices. As we noticed, this is usually performed through 
the development of methods which automate tasks like test case creation and enhance different requirement analysis procedures.\\

Regarding the input type characteristic, we frequently encountered approaches that operate given documents containing natural language requirements and specifications about the tests to be developed. 
These documents act as a base for the whole testing process since they describe necessary prerequisites of the system under test. They also contain the expected functionality of the product based on which 
it is later going to be tested for. Consequently, regardless of the target of the proposed approach, requirement documents are an important source of information for software testing, which is sometimes 
difficult to make use of because of the fact that they contain natural language text that is not a very firendly data format for the computer to process. However, thanks to NLP, that kind of data types 
are being now more frequently and easily utilized.\\

Moving on to the output type, we find that test case code is one of the main data types generated by the proposed approaches. This finding aligns with the fact that many of the papers focus on the implementation 
stage, as we refered earlier. Thus, the test case generation process creates executable test code for the system under test.\\

The last criterion based on which we analyzed the papers is the NLP Techniques used in the approaches. We separated the NLP techniques in NLU and NLG methods and discussed them individually. What we eventually 
ended up on, is that the majority of the papers perform NLU tasks and use popular methods to achieve that. A widely and well-known technique used during Lexical Analysis is tagging and, more specifically, POS Tagging. 
POS Tagging identifies the part of speech to which the given words belong. However, apart from the classic use of POS Tagging, we identified several cases to which different customizations have been applied to the method. 
Those customizations are related to the choice of different tags than the classic parts of speech \cite{9240680}, whereas another paper uses BIO Tagging to further facilitate the process \cite{pedemonte2012towards}. 
Other authors even proceeded to create their own customized POS Tagger \cite{carvalho2014nat2testscr}.\\
Moreover, the use of NLP tools and libraries is also pretty often. Two distinct ones we noted are the Stanford CoreNLP library and the Python NLTK. Both of them are mainly used to support NLP parsing tasks, even though 
they provide a variety of functionalities. Word2Vec is another popular tool which turned out to be pretty popular among the papers reviewed. This tool facilitates the process of transforming the semantic representation of 
words into vectors, a task necessary in order to utilize word embeddings. To conclude, several statistical measures were also used by the authors with some of them being widely known in NLP community. The metrics applied 
in most of the papers are TF-IDF and cosine similarity. Both of them provide an indication of the semantic similarity between two elements using their vector representations.\\
Overall, we notice a tendency to continue using long-established popular NLP methods, and customizing them if necessary to achieve the requested goal. However, this doesn't mean that attempts to incorporate more unusual 
practices have not been made. LSTM RNNs and Relaxed Word-Mover's Distance are two examples of methods which we saw being used in a low frequency among the approaches. We believe, though, that the desired complexity 
of the technique and the quality of the result play an important role in the type of method which will be used during development. During this review, we discovered that it is not necessary to use complex methods in 
order to achieve substantial results and make a significant contribution to this knowledge area.

\chapter{Conclusion and Future Work}

This thesis paper classified the state-of-the-art and the practices of NLP used in the Software Testing process. We reviewed 44 scientific papers written after the year 2000 that focus on this 
knowledge area and contribute to different testing tasks. This Systematic Literature Review organizes the already existing knowledge of the scientific literature and it is a well structured source of 
information for both researchers and practitioners who operate in the software testing field.\\

Also, the incorporation of a popular Machine Learning process --- NLP --- in this knowledge area increases the level of alignment of this review with current practice and trends. By creating this study, 
we aim to make current practices and upcoming needs accessible to anyone interested. We simplified the searching process by providing recent organized information which can be used as a starting point 
for further research.\\

Given that, the present review motivates the business community to utilize outcomes and results of this paper in order to express or verify current needs and challenges faced in technical processes during 
practice. The scientific community is encouraged to use this study and the above expressed needs and challenges as a source of information in order to contribute to the solution of those problems.

