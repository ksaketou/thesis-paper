\chapter{Χαρακτηριστικά Πρακτικής Άσκησης}

\section{Software Testing Services Center}

Το Software Testing Services Center στο οποίο εντάχθηκα σκοπεύει στην διασφάλιση της ποιότητας των προϊόντων της εταιρείας αλλά και των πελατών της. Είναι υπεύθυνο για τον έλεγχο του εάν οι εφαρμογές λειτουργούν με τον αναμενόμενο και πιο αποδοτικό τρόπο και αυτό επιτυγχάνεται μέσω της εφαρμογής πολυάριθμων ειδών ελέγχων στα αντίστοιχα λογισμικά. Τα είδη των ελέγχων αυτών παραθέτονται παρακάτω:\\

\subsection*{Performance Testing}
Σκοπός του συγκεκριμένου είδους ελέγχου είναι να εξασφαλίσει την ταχύτητα, επεκτασιμότητα και σταθερότητα του λογισμικού. Συγκεκριμένα, δίνει έμφαση σε μετρικές όπως ο χρόνος απόκρισης της εφαρμογής, η αξιοπιστία, ο τρόπος με τον οποίο χρησιμοποιούνται οι πόροι αποσκοπώντας στην όσο το δυνατόν πιο αποδοτική τους αξιοποίηση. Το Performance Testing αποτελείται από τα τρία παρακάτω είδη ελέγχων.

\begin{itemize}
    \item Load Testing\\
    Στόχος του Load Testing είναι να ελέγξει το πώς ανταποκρίνεται μια εφαρμογή, ένα σύστημα σε μεγάλο αριθμό ταυτόχρονων χρηστών, οι οποίοι είναι ενεργοι και εκτελούν διάφορες ενέργειες σε αυτό.
    \item Volume Testing\\
    Όταν εφαρμόζουμε Volume Testing, παρέχουμε στην εφαρμογή μας ένα πολύ μεγάλο όγκο δεδομένων. Στόχος είναι να διαπιστωθεί έαν το σύστημα λειτουργεί όπως αναμένεται, ακόμη και με έναν μεγάλο όγκο δεδομένων. Παρά το γεγονός ότι αυτό το είδος ελέγχου μοιάζει με το Load Testing, όταν εφαρμόζουμε το δεύτερο, θέλουμε να δούμε το πώς μεταβάλλεται και επηρεάζεται η απόδοση της εφαρμογής μας όταν αυξάνεται ο όγκος δεδομένων, χωρίς να δίνουμε βάση στον τρόπο εκτέλεσης των αναμενόμενων λειτουργιών.
    \item Stress Testing\\
    Στόχος του Stress Testing είναι να εντοπίσει το ανώτατο όριο φόρτου (χρηστών, πόρων κλπ.) στο οποίο η εφαρμογή μπορεί να ανταποκριθεί αποτελεσματικά. Ακόμη, βάση δίνεται και στο πόσο ομαλή θα είναι η επαναφορά του συστήματος στο αναμενόμενο φόρτο πόρων μετά από ένα πιο απαιτητικό χρονικό διάστημα όσον αφορά την απαιτούμενη απόκριση.
\end{itemize}

\subsection*{GUI Testing}
Σε αυτό το είδος ελέγχου, στόχος είναι να διασφαλιστεί ότι η γραφική διεπαφή του χρήστη είναι σχεδιασμένη σύμφωνα με τις απαιτήσεις και οδηγίες 
του πελάτη και πως μέσω αυτής εκτελούνται αποτελεσματικά όλες οι λειτουργίες της εφαρμογής. Συγκεκριμένα, ελέγχεται η εμφάνιση των οθονών, τα 
διάφορα κουμπιά, τα πεδία εισόδου δεδομένων (checkboxes, radio buttons, text input fields κλπ.), καθώς επίσης και η λειτουργικότητά τους.

\subsection*{API Testing}
Κατά το συγκεκριμένο είδος ελέγχου, διασφαλίζονται διάφορες μετρικές του συστήματός μας όπως απόδοση, λειτουργικότητα κλπ. Τα APIs δεν 
περιέχουν διεπαφή χρήστη, οπότε όλος ο έλεγχος γίνεται σε επίπεδο διαδικτυακών συναλλαγών μηνυμάτων (web transactions) μεταξύ web client και 
web server. Γενικότερα το API Testing εξυπηρετεί την μεθοδολογία Agile Software Development, καθώς προσαρμόζεται εύκολα σε νέα λειτουργικότητα, 
σε αντίθεση με τους ελέγχους μέσω γραφικής διεπαφής, οι οποίοι απαιτούν περισσότερο χρόνο για να συντηρηθούν.

\subsection*{Data Migration Testing}
Αυτός ο τύπος ελέγχου εφαρμόζεται στην περίπτωση που μια εφαρμογή ή ένα σύστημα πρόκειται να μεταφερθεί σε μια νέα υποδομή. Κατά το Migration 
Testing, διασφαλίζεται πως η εφαρμογή μεταβαίνει στη νέα υποδομή χωρίς να υπάρχει κάποια συνέπεια στην ακεραιότητα των δεδομένων. Επίσης, 
επιβεβαιώνεται πως δεν υπήρξε απώλεια δεδομένων κατά τη μεταφορά.

\subsection*{Βασικές Διαδικασίες Τμήματος}
Το Software Testing Services Center είναι υπεύθυνο για τον έλεγχο της λειτουργικότητας και απόδοσης των διαφόρων εφαρμογών που ανήκουν και χρησιμοποιούνται από τους 
Οργανισμούς και πελάτες της Netcompany-Intrasoft. Η εταιρεία μπορεί να αναλάβει να τεστάρει τόσο εφαρμογές τις οποίες αναπτύσσει εκείνη όσο και εφαρμογές που αναπτύσσονται 
από τους ίδιους τους πελάτες. Και στις δύο περιπτώσεις βασική προϋπόθεση αποτελεί αρχικά η μελέτη και ανάλυση των λειτουργικών και μη λειτουργικών απαιτήσεων του συστήματος. 
Οι λειτουργικές απαιτήσεις βοηθούν στην κατανόηση των δυνατοτήτων της εφαρμογής, ενώ οι μη λειτουργικές αποτελούν σημαντική πληροφορία για είδη ελέγχων όπως το Performance Testing 
καθώς επίσης οριοθετούν κατά ένα βαθμό την εφαρμογή σε ότι αφορά τις δυνατότητές της. Στη συνέχεια, ακολουθεί το Test Planning κατά το οποίο προγραμματίζεται η όλη διαδικασία του 
testing λαμβάνοντας υπόψη τους εκτιμώμενους χρόνους σχεδιασμού, υλοποίησης και εκτέλεσης των test cases, οι οποίοι πρέπει να συμβαδίζουν με τις αντίστοιχες προθεσμίες παράδοσης 
των ελέγχων που έχουν ζητηθεί από τον πελάτη. Συνεπώς, το Test Planning αποτελεί μια ιδιαίτερα σημαντική αλλά και κρίσιμη διαδικασία του Testing Lifecycle. Έπειτα το τμήμα προχωράει 
στον σχεδιασμό των test cases, βεβαιώνοντας ότι ικανοποιούνται όλα τα ζητούμεα κριτήρια ελέγχων που έχουν καθοριστεί από τους stakeholders. Μόλις ολοκληρωθεί ο σχεδιασμός, ακολουθεί η 
συγγραφή των automated test scripts με στόχο την υλοποίηση των test cases. Τέλος, έπεται η διαδικασία του Reporting κατά την οποία παραθέτονται τα αποτελέσματα των ελέγχων της εφαρμογής.\\

Η παραπάνω διαδικασία αποτελεί των κορμό των λειτουργιών που εκτελεί το Software Testing Services Center. Ακόμη, ακολουθείται η Agile μεθοδολογία ολοκλήρωσης των διαδικασιών με αυτό να 
συνεπάγεται συχνά παραδοτέα και ενημέρωση προς τους πελάτες. Έτσι, ο συνεχής συντονισμός και προγραμματισμός των διαδικασιών είναι κάτι απαραίτητο που συμβαίνει ανάμεσα στα μέλη του τμήματος 
μέσω αποτελεσματικής επικοινωνίας και συζήτησης.

\section*{Θέση Πρακτικής Άσκησης}
Όπως αναφέρθηκε και παραπάνω, το Software Testing Services Center αποτελείται από Test Analysts και Test Engineers. Εγώ ακολούθησα το μονοπάτι του Test Engineer, λαμβάνοντας συγκεκριμένα τον ρόλο του Test Automation Engineer, του οποίου τα καθήκοντα περιγράφονται παρακάτω.\\

\begin{itemize}
    \item Ανάλυση των σεναρίων ελέγχου, μελέτη και διερεύνηση των λειτουργικών και μη λειτουργικών απαιτήσεων του συστήματος που πρόκειται να υποβληθούν σε έλεγχο.
    \item Ανάλυση του τι αξίζει να αυτοματοποιηθεί, μελέτη των σεναρίων ελέγχου και καθορισμός αυτών που χρίζουν αυτοματοποίησης. Τα υπόλοιπα σενάρια ελέγχου εκτελούνται manually από τους Manual Testers.
    \item Σχεδιασμός και προετοιμασία των αυτοματοποιημένων ελέγχων και δημιουργία των απαραίτητων δεδομένων ελέγχου.
    \item Οργάνωση των test cycles, εκτίμηση του απαιτούμενου χρόνου υλοποίησης και ολοκλήρωσης των καθορισμένων ελέγχων για να παραδοθεί έγκαιρα η αντίστοιχη αναφορά στους stakeholders.
    \item Προετοιμασία των στατιστικών δεδομένων από την εκτέλεση των αυτοματοποιημένων ελέγχων της εφαρμογής. Στη συνέχεια, τα δεδομένα αυτά ενσωματόνονται και υποβάλλονται στους stakeholders.
\end{itemize}

\section{Μεθοδολογίες και Εργαλεία}

Η ομάδα του Software Testing Services Center χρησιμοποιεί έναν μεγάλο αριθμό εργαλείων, μεθοδολογιών και τεχνολογιών για την εκτέλεση 
των λειτουργιών της. Σε αυτό το μέρος γίνεται μια αναφορά σε όλα τα παραπάνω.

\section*{Μεθοδολογίες}
\subsection*{Behavior Driven Development}
Το Behavior Driven Development αποτελεί μια μεθοδολογία η οποία καθορίζει ουσιαστικά την συμπεριφορά του υπο ανάπτυξη συστήματος. Σύμφωνα με 
τον Smart \cite{bddbook}, η μεθοδολογία αυτή μπορεί να εφαρμοστεί τόσο σε υψηλό όσο και σε χαμηλό επίπεδο. Το υψηλό επίπεδο αναφέρεται στην περιγραφή 
των λειτουργικών απαιτήσεων του συστήματος δίνοντας έμφαση περισσότερο στο επιχειρησιακό κομμάτι. Όσον αφορά το χαμηλό επίπεδο, αυτό εστιάζει 
στην περιγραφή της συμπεριφοράς της εφαρμογής βάσει εκτελέσιμων προδιαγραφών. Ακολουθώντας αυτή την πρακτική, ο κώδικας που γράφεται είναι πιο 
εύκολα κατανοητός και συντηρήσιμος. 

\subsection*{Page Object Pattern}
Ο σχεδιασμός των test cases με τη χρήση των page objects ομαδοποιεί κατά έναν τρόπο όλα τα στοιχεία και τη λειτουργικότητα που υπάρχει σε μια 
συγκεκριμένη οθόνη μιας ιστοσελίδας και τα συγκεντρώνει σε ένα ενιαίο σημείο/κλάση. Με αυτό τον τρόπο, τυχόν αλλαγές που ίσως προκύψουν στην αντίστοιχη 
οθόνη θα απαιτούν μόνο αλλαγές στην αντίστοιχη κλάση και όχι σε πολλά σημεία του κώδικα ταυτόχρονα. Έτσι, υπάρχει ξεκάθαρος διαχωρισμός μεταξύ των στοιχείων 
που περιέχει κάθε οθόνη της εφαρμογής που ελέγχεται (πχ. κουμπιά, εικόνες κ.ά.), χωρίς να περιπλέκεται η δομή του κώδικα. Ακόμη, η λειτουργικότητα 
που προσφέρει κάθε σημείο της εφαρμογής είναι ομαδοποιημένη και πολύ πιο εύκολο να εντοπιστεί από τον αντίστοιχο tester στον κώδικα \cite{pageobject}.

\section*{Εργαλεία}
\subsection*{Java}
Η γλώσσα προγραμματισμού που χρησιμοποιείται σε αρκετά μεγάλο βαθμό για τον έλεγχο των συστημάτων και των εφαρμογών είναι η Java. Αποτελεί μια 
αντικειμενοστραφής γλώσσα, βασίζεται, δηλαδή, σε κλάσεις και αντικείμενα. Χάρη στις πολυάριθμες δυνατότητες και εφαρμογές της, δίνει την 
δυνατότητα εκτέλεσης πολλών περίπλοκων λειτουργιών και ανταποκρίνεται, έτσι, στις απαιτήσεις του τμήματος.
\subsection*{Maven}
To Maven αποτελεί ένα εργαλείο αυτοματοποίησης συνήθως διαφόρων projects γραμμένων σε Java. Εξυπηρετεί στην οργάνωση και την αποτελεσματικότερη 
διαχείριση της δομής τους. Ακόμη, διευκολύνει διάφορα στάδια του Development Life Cycle με το testing να είναι ένα από αυτά.
\subsection*{Git}
To Git αποτελεί ένα Version Control Tool, το οποίο επιτρέπει την ευκολότερη διαχείριση διαφόρων σταδίων ανάπτυξης ενός project. Εξυπηρετεί 
στον ευκολότερο εντοπισμό των αλλαγών που συμβαίνουν, διευκολύνοντας την ομαδική συνεργασία και ανάπτυξη κώδικα. Η ομάδα χρησιμοποιεί το GitLab 
για την διαχείριση των repositories.
\subsection*{Selenium}
Το εργαλείο αυτό είναι ένα από τα πιο βασικά εργαλεία που χρησιμοποιεί η ομάδα. Προσφέρει αυτοματοποίηση των φυλλομετρητών και χρησιμοποιείται 
έντονα στον τομέα του Automation Testing. Μέσω των βιβλιοθηκών που παρέχει, επιτυγχάνεται η ευκολότερη και γρήγορη εκτέλεση αυτοματοποιημένων 
ελέγχων.
\subsection*{Cucumber}
Το Cucumber αποτελεί ένα εργαλείο που υποστηρίζει το Behavior Driven Development - BDD που αναφέρθηκε και παραπάνω. Χάρη στη γλώσσα Gherkin, η 
αναμενόμενη συμπεριφορά του λογισμικού εκφράζεται με πολύ απλό και κατανοητό τρόπο, ο οποίος πλησιάζει αρκετά την φυσική γλώσσα. Έτσι, γίνεται 
εύκολα κατανοητή από τον οποιονδήποτε και, συνεπώς, από τους πελάτες.
\subsection*{REST-Assured API}
Το συγκεκριμένο εργαλείο παρέχεται από την Java και εξυπηρετεί στο αυτοματοποιημένο API Testing. Συγκεκριμένα, εξυπηρετεί στον έλεγχο των REST 
Services, μέσω HTTP Requests.
\subsection*{SoapUI}
Το SoapUI αποτελεί και αυτό ένα εργαλείο για Web Service Testing. Παρ' όλα αυτά, μπορεί να χρησιμοποιηθεί και για Functional και Load Testing.
\subsection*{JMeter}
Το JMeter αποτελεί ένα εργαλείο, το οποίο χρησιμοποιείται για Performance Testing. Παρέχει, δηλαδή, στην εφαρμογή μας έναν μεγάλο όγκο 
δεδομένων με στόχο την καταγραφή του πώς αυτό ανταποκρίνεται και το πώς αποδίδει.