\chapter*{Χαρακτηριστικά Πρακτικής Άσκησης}

\section*{\textlatin{Software Testing Services Center}}

Το \textlatin{Software Testing Services Center} στο οποίο εντάχθηκα σκοπέυει στην διασφάλιση της ποιότητας των προϊόντων της εταιρείας αλλά και των πελατών της. Είναι υπεύθυνο για τον έλεγχο του εάν οι εφαρμογές λειτουργούν με τον αναμενόμενο και πιο αποδοτικό τρόπο και αυτό επιτυγχάνεται μέσω της εφαρμογής πολυάριθμων ειδών ελέγχων στα αντίστοιχα λογισμικά. Τα είδη των ελέγχων αυτών παραθέτονται παρακάτω:\\

\textlatin{\textbf{Performance Testing}}\\
Σκοπός του συγκεκριμένου είδους ελέγχου είναι να εξασφαλίσει την ταχύτητα, επεκτασιμότητα και σταθερότητα του λογισμικού. Συγκεκριμένα, δίνει έμφαση σε μετρικές όπως ο χρόνος απόκρισης της εφαρμογής, η αξιοπιστία, ο τρόπος με τον οποίο χρησιμοποιούνται οι πόροι αποσκοπώντας στην όσο το δυνατόν πιο αποδοτική τους αξιοποίηση. Το \textlatin{Performance Testing} αποτελείται από τα τρία παρακάτω είδη ελέγχων.

\begin{itemize}
    \item \textlatin{Load Testing}\\
    Στόχος του \textlatin{Load Testing} είναι να ελέγξει το πώς ανταποκρίνεται μια εφαρμογή, ένα σύστημα σε μεγάλο αριθμό ταυτόχρονων χρηστών, οι οποίοι είναι ενεργοι και εκτελούν διάφορες ενέργειες σε αυτό.
    \item \textlatin{Volume Testing}\\
    Όταν εφαρμόζουμε \textlatin{Volume Testing}, παρέχουμε στην εφαρμογή μας ένα πολύ μεγάλο όγκο δεδομένων. Στόχος είναι να διαπιστωθεί έαν το σύστημα λειτουργεί όπως αναμένεται, ακόμη και με έναν μεγάλο όγκο δεδομένων. Παρά το γεγονός ότι αυτό το είδος ελέγχου μοιάζει με το \textlatin{Load Testing}, όταν εφαρμόζουμε το δεύτερο, θέλουμε να δούμε το πώς μεταβάλλεται και επηρεάζεται η απόδοση της εφαρμογής μας όταν αυξάνεται ο όγκος δεδομένων, χωρίς να δίνουμε βάση στον τρόπο εκτέλεσης των αναμενόμενων λειτουργιών.
    \item \textlatin{Stress Testing}\\
    Στόχος του \textlatin{Stress Testing} είναι να εντοπίσει το ανώτατο όριο φόρτου (χρηστών, πόρων κλπ.) στο οποίο η εφαρμογή μπορεί να ανταποκριθεί αποτελεσματικά. Ακόμη, βάση δίνεται και στο πόσο ομαλή θα είναι η επαναφορά του συστήματος στο αναμενόμενο φόρτο πόρων μετά από ένα πιο απαιτητικό χρονικό διάστημα όσον αφορά την απαιτούμενη απόκριση. \\
\end{itemize}

\textlatin{\textbf{GUI Testing}}\\
Σε αυτό το είδος ελέγχου, στόχος είναι να διασφαλίσει ότι η γραφική διεπαφή του χρήστη είναι σχεδιασμένη σύμφωνα με τις απαιτήσεις και οδηγίες του πελάτη και πως μέσω αυτής εκτελούνται αποτελεσματικά όλες οι λειτουργίες της εφαρμογής. Συγκεκριμένα, ελέγχεται η εμφάνιση των οθονών, τα διάφορα κουμπιά, τα πεδία εισόδου δεδομένων (\textlatin{checkboxes, radio buttons, text input fields} κλπ.), καθώς επίσης και η λειτουργικότητά τους.\\

\textlatin{\textbf{API Testing}}\\
Κατά το συγκεκριμένο είδος ελέγχου, διασφαλίζονται διάφορες μετρικές του συστήματός μας όπως απόδοση, λειτουργικότητα κλπ. Τα \textlatin{APIs} δεν περιέχουν διεπαφή χρήστη, οπότε όλος ο έλεγχος γίνεται σε επίπεδο διαδικτυακών συναλλαγών μηνυμάτων (\textlatin{web transactions}) μεταξύ \textlatin{web client} και \textlatin{web server}. Γενικότερα το \textlatin{API Testing} εξυπηρετεί την μεθοδολογία \textlatin{Agile Software Development}, καθώς προσαρμόζεται εύκολα σε νέα λειτουργικότητα, σε αντίθεση με τους ελέγχους μέσω γραφικής διεπαφής, οι οποίοι απαιτούν περισσότερο χρόνο για να συντηρηθούν.\\

\textlatin{\textbf{Data Migration Testing}}\\
Αυτός ο τύπος ελέγχου εφαρμόζεται στην περίπτωση που μια εφαρμογή ή ένα σύστημα πρόκειται να μεταφερθεί σε μια νέα υποδομή. Κατά το \textlatin{Migration Testing}, διασφαλίζεται πως η εφαρμογή μεταβαίνει στη νέα υποδομή χωρίς να υπάρχει κάποια συνέπεια στην ακεραιότητα των δεδομένων. Επίσης, επιβεβαιώνεται πως δεν υπήρξε απώλεια δεδομένων κατά τη μεταφορά.\\

\section*{Θέση Πρακτικής Άσκησης}
Όπως αναφέρθηκε και παραπάνω, το \textlatin{Software Testing Services Center} αποτελείται από \textlatin{Test Analysts} και \textlatin{Test Engineers}. Εγώ ακολούθησα το μονοπάτι του \textlatin{Test Engineer}, λαμβάνοντας συγκεκριμένα τον ρόλο του \textlatin{Test Automation Engineer}, του οποίου τα καθήκοντα περιγράφονται παρακάτω.\\

\begin{itemize}
    \item Ανάλυση των σεναρίων ελέγχου, μελέτη και διερεύνηση των λειτουργικών και μη λειτουργικών απαιτήσεων του συστήματος που πρόκείται να υποβλγθούν σε έλεγχο.
    \item Ανάλυση του τι αξίζει να αυτοματοποιηθεί, μελέτη των σεναρίων ελέγχου και καθορισμός αυτών που χρίζουν αυτοματοποίησης. Τα υπόλοιπα σενάρια ελέγχου εκτελούνται \textlatin{manually} από τους \textlatin{Manual Testers}.
    \item Σχεδιασμός και προετοιμασία των αυτοματοποιημένων ελέγχων και δημιουργία των απαραίτητων δεδομένων ελέγχου.
    \item Οργάνωση των \textlatin{test cycles}, εκτίμηση του απαιτούμενου χρόνου υλοποίησης και ολοκλήρωσης των καθορισμένων ελέγχων για να παραδοθεί έγκαιρα η αντίστοιχη αναφορά στους \textlatin{stakeholders}.
    \item Προετοιμασία των στατιστικών δεδομένων από την εκτέλεση των αυτοματοποιημένων ελέγχων της εφαρμογής. Στη συνέχεια, τα δεδομένα αυτά ενσωματόνονται και υποβάλονται στους \textlatin{stakeholders}.
\end{itemize}

\section*{Μεθοδολογίες και Εργαλεία}

Η ομάδα του \textlatin{Software Testing Services Center} χρησιμοποιεί έναν μεγάλο αριθμό εργαλείων, μεθοδολογιών και τεχνολογιών για την εκτέλεση των λειτουργιών του. Σε αυτό το μέρος γίνεται μια αναφορά σε όλα τα παραπάνω.\\

\section*{Μεθοδολογίες}
\textbf{\textlatin{Behavior Driven Development}}\\
κείμενο\\ \\
\textbf{\textlatin{Page Object Pattern}}\\
κείμενο\\
\section*{Εργαλεία}

\textbf{\textlatin{Java}}\\
Η γλώσσα προγραμματισμού που χρησιμοποιείται σε αρκετά μεγάλο βαθμό για τον έλεγχο των συστημάτων και των εφαρμογών είναι η \textlatin{Java}. Αποτελεί μια αντικειμενοστραφής γλώσσα, βασίζεται, δηλαδή, σε κλάσεις και αντικείμενα. Χάρη στις πολυάριθμες δυνατότητες και εφαρμογές της, δίνει την δυνατότητα εκτέλεσης πολλών περίπλοκων λειτουργιών και ανταποκρίνεται, έτσι, στις απαιτήσεις του τμήματος.\\ \\
\textbf{\textlatin{Maven}}\\
To \textlatin{Maven} αποτελεί ένα εργαλείο αυτοματοποίησης συνήθως διαφόρων \textlatin{projects} γραμμένων σε \textlatin{Java}. Εξυπηρετεί στην οργάνωση και την αποτελεσματικότερη διαχείριση της δομής τους. Ακόμη, διευκολύνει διάφορα στάδια του \textlatin{Development Life Cycle} με το \textlatin{testing} να είναι ένα από αυτά.\\ \\
\textbf{\textlatin{Git}}\\
To \textlatin{Git} αποτελεί ένα \textlatin{Version Control Tool}, το οποίο επιτρέπει την ευκολότερη διαχείριση διαφόρων σταδίων ανάπτυξης ενός \textlatin{project}. Εξυπηρετεί στον ευκολότερο εντοπισμό των αλλαγών που συμβαίνουν, διευκολύνοντας την ομαδική συνεργασία και ανάπτυξη κώδικα. Η ομάδα χρησιμοποιεί το \textlatin{GitLab} για την διαχείριση των \textlatin{repositories}.\\ \\
\textbf{\textlatin{Selenium}}\\
Το εργαλείο αυτό είναι ένα από τα πιο βασικά εργαλεία που χρησιμοποιεί η ομάδα. Προσφέρει αυτοματοποίηση των φυλλομετρητών και χρησιμποιείται έντονα στον τομέα του \textlatin{Automation Testing}. Μέσω των βιβλιοθηκών που παρέχει, επιτυγχάνεται η ευκολότερη και γρήγορη εκτέλεση αυτοματοποιημένων ελέγχων.\\ \\
\textbf{\textlatin{Cucumber}}\\
Το \textlatin{Cucumber} αποτελεί ένα εργαλείο που υποστηρίζει το \textlatin{Behavior Driven Development - BDD} που αναφέρθηκε και παραπάνω. Χάρη στη γλώσσα \textlatin{Gherkin}, η αναμενόμενη συμπεριφορά του λογισμικού εκφράζεται με πολύ απλό και κατανοητό τρόπο, ο οποίος πλησιάζει αρκετά την φυσική γλώσσα. Έτσι, γίνεται εύκολα κατανοητή από τον οποιονδήποτε και, συνεπώς, από τους πελάτες. \\ \\
\textbf{\textlatin{REST-Assured API}}\\
Το συγκεκριμένο εργαλείο παρέχεται από την \textlatin{Java} και εξυπηρετεί στο αυτοματοποιημένο \textlatin{API Testing}. Συγκεκριμένα, εξυπηρετεί στον έλεγχο των \textlatin{REST Services}, μέσω \textlatin{HTTP Requests}.\\ \\ 
\textbf{\textlatin{SoapUI}}\\
Το \textlatin{SoapUI} αποτελεί και αυτό ένα εργαλείο για \textlatin{Web Service Testing}. Παρ' όλα αυτά, μπορεί να χρησιμοποιηθεί και για \textlatin{Functional} και \textlatin{Load Testing}.\\ \\
\textbf{\textlatin{JMeter}}\\
Το \textlatin{JMeter} αποτελεί ένα εργαλείο, το οποίο χρησιμοποιείται για \textlatin{Performance Testing}. Παρέχει, δηλαδή, στην εφαρμογή μας έναν μεγάλο όγκο δεδομένων με στόχο την καταγραφή του πώς αυτό ανταποκρίνεται και το πώς αποδίδει.\\ \\