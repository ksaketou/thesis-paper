\chapter*{Έργα και Δραστηριότητες}

\section*{Εισαγωγική Εκπαίδευση}
Οι δύο πρώτες εβδομάδες μετά από την έναρξη της πρακτικής μου άσκησης περιείχαν την διαδικασία εκπαίδευσής 
μου πάνω στα εργαλεία, τις μεθοδολογίες και τον τρόπο με τον οποίο εργάζεται το Software Testing Services 
Center. Η εκπαίδευση γινόταν μέσω της παρακολούθησης καταγεγραμμένων training sessions της ομάδας μου τα 
οποία  είχαν πραγματοποιηθεί στο παρελθόν και γίνονταν από τον team leader της ομάδας μου. Παράλληλα με τα 
sessions αυτά, είχα και πρακτική επαφή με το αντικείμενο πάνω στο οποίο εκπαιδευόμουν καθώς έφτιαχνα τα μικρά
 demo projects πάνω στα οποία γινόταν η εκάστοτε εκπαίδευση. Αυτά τα demo projects στη συνέχεια γινόντουσαν 
 push στο Gitlab της εταιρείας. Η δομή του «Test Engineer Learning Path» το οποίο παρακολούθησα για την 
 εκπαίδευσή μου περιείχε τα εργαλεία που χρησιμοποιεί η ομάδα, τα οποία προαναφέρθηκαν στην προηγούμενη ενότητα. \\

 Οι δυσκολίες που αντιμετώπισα κατά την διάρκεια της εκπαίδευσης αποτελούσαν κυρίως τεχνικά θέματα, όπως η πρόσβαση σε κάποιο 
 εργαλείο, και bugs τα οποία συναντούσα κατά την υλοποίηση των demo projects. Το θέμα της πρόσβασης σε κάποιο λογισμικό λυνόταν 
 σύντομα με την βοήθεια του IT Support, με τους οποίους επικοινωνούσα όταν είχα κάποιο αντίστοιχο πρόβλημα. Όσον αφορά τεχνικά θέματα 
 πάνω στα projects και τα bugs, πολλές φορές τα έλυνα εγώ, ενώ υπήρχαν περιπτώσεις στις οποίες λάμβανα βοήθεια από άλλο μέλος της ομάδας μου.

Η Εισαγωγική Εκπαίδευση διήρκησε από 21/3/2022 έως 1/4/2022.

\section*{UI Test Case Debugging - SDMXRi}
Μετά την εισαγωγική εκπαίδευση, εντάχθηκα στα projects της Eurostat ξεκινώντας με το debugging των test cases που σχετίζονταν με το User 
Interface της Web Εφαρμογής SDMX-RI. \\

«\emph{The SDMX-Reference Infrastructure (SDMX-RI) is a set of pick-and-choose building blocks and tools that allow statistical data to be 
exposed to the external world through access rights by using web services.}» 
\begin{flushright}  
    \href{https://ec.europa.eu/eurostat/cros/content/sdmx-ri_en}{- European Commission} \\
\end{flushright}

Συγκεκριμένα, η διαδικασία αποτελούνταν από την εκτέλεση ενός αυτοματοποιημένου test case, το οποίο άνοιγε την εφαρμογή στον browser και 
εκτελούσε αυτόνομα όλα τα βήματα που περιγράφονταν, ακριβώς όπως θα τα έκανε κάποιος πραγματικός χρήστης. Ωστόσο, επειδή η γραφική διεπαφή 
της εφαρμογής είχε πλέον υποστεί κάποιες αλλαγές, όπως μορφή/θέση κουμπιών, προσθήκη/αφαίρεση παραθύρων με ενημερωτικά μηνύματα κλπ, αρκετές 
από αυτές τις περιπτώσεις ελέγχου δεν εκτελούνταν επιτυχώς. Συνεπώς, μου ανατέθηκε η προσαρμογή των test cases αυτών στην νέα version της 
εφαρμογής (v 6.17.2). Οι αλλαγές που χρειάστηκε να κάνω στα test cases ήταν των παρακάτω ειδών:

\begin{itemize}
    \item Τροποποίηση των Selenium locators (XPath, CssSelector κλπ) ορισμένων elements (κουμπί, σύνδεσμος, dropdown κλπ.)
    \item Προσθήκη/αφαίρεση βημάτων (κλικ σε elements, αποδοχή/απόρριψη modals της εφαρμογής και του browser κλπ.)
    \item Προσαρμογή των δεδομένων εισόδου των test cases 
\end{itemize}

Κατά τη διάρκεια του debugging, ορισμένα test cases δεν εκτελούνταν σωστά και ο λόγος για τον οποίο συνέβαινε αυτό φαίνεται να οφειλόταν 
στην λειτουργία της εφαρμογής και όχι στα test cases. Στην περίπτωση που το πρόβλημα οφείλεται στην εφαρμογή, η ομάδα του testing οφείλει 
να ενημερώσει την development ομάδα για το bug, έτσι ώστε να το διορθώσει. Αυτό συνέβη δύο φορές κατά την διάρκεια της δραστηριότητας. 
Την πρώτη φορά, εγώ και δυο άλλα μέλη της ομάδας μου διοργανώσαμε ένα meeting για την ενημέρωση ενός από τα μέλη της development ομάδας 
για το bug. Στην δεύτερη περίπτωση, η ενημέρωση έγινε από εμένα προσωπικά στο ίδιο πάλι άτομο της development ομάδας.
Εκτός από τα παραπάνω bugs, προέκυψε και ένα πρόβλημα με το set up του Database Environment της εφαρμογής. Έτσι, διοργανώθηκε άλλο ένα 
meeting μαζί με δυο μέλη της ομάδας μου και ένα άτομο της development ομάδας, το οποίο θα μας βοηθούσε να  αντιμετωπίσουμε το πρόβλημα, όπως 
και εν τέλει έγινε.

Το παραδοτέο της παραπάνω δραστηριότητας είναι οι αλλαγές στον κώδικα που έκανα, οι οποίες αποτέλεσαν το «develop-6.17.2» branch του project 
καθώς επίσης και ο φάκελος με τα test reports από την εκτέλεση των τεστ. Η δραστηριότητα διήρκησε από τις 4/4/2022 έως τις 8/4/2022.

\section*{Δημιουργία Προσχεδίων για τα Test Cases των νέων features της εφαρμογής SDMX-Ri}
Στην συγκεκριμένη δραστηριότητα, μου ανατέθηκε να δημιουργήσω προσχέδια των test cases για την λειτουργικότητα που προστέθηκε στο SDMX-RI στην 
version 6.17. Η διαδικασία αυτή θα εξυπηρετήσει, στην συνέχεια, στο να έχει ήδη φτιαχτεί μια βάση για τα νέα test cases μόλις παραδοθούν από 
την development ομάδα στο testing για υλοποίηση. Η νέα λειτουργικότητα αναφέρεται παρακάτω:
\begin{itemize}
    \item Monitor Users (Settings): διαχείριση από τον admin της δραστηριότητας των καθορισμένων χρηστών
    \item Nsi Web Service Endpoints (Settings): διαχείριση των endpoints που αφορούν το Nsi web client
	\item Logger System (Settings): διαχείριση και προβολή των logs των http requests από/προς την εφαρμογή
	\item Browse and Download Data (Dataset menu option): περιήγηση και πρόσβαση σε στατιστικά dataset
\end{itemize}
Λόγω του ότι δεν είχαμε λάβει ακόμα πληροφορίες από την development ομάδα σχετικά με τα test cases των νέων λειτουργιών, όπως τις απαιτήσεις 
τους και τα δεδομένα εισόδου, δεν υπήρχε μεγάλο περιθώριο υλοποίησής τους. Συνεπώς, υλοποίησα κάποια πολύ αρχικά στάδια (π.χ. έλεγχος πρόσβασης 
του χρήστη στην αντίστοιχη οθόνη). Επιπλέον, η εταιρία υλοποιεί τον κώδικά της βασιζόμενη στο Page Object Model, σύμφωνα με το οποίο κάθε 
κλάση/πακέτο κλάσεων αντιπροσωπεύει μια οθόνη της εφαρμογής. Υλοποίησα, λοιπόν, και τις αντίστοιχες κλάσεις για τη νέα λειτουργικότητα σύμφωνα 
με το μοντέλο αυτό. Η δραστηριότητα διήρκησε από τις 12/4/2022 έως τις 13/4/2022.

\section*{UI Test Case Troubleshooting/Debugging – ESDEN}
Στην δραστηριότητα αυτή ασχολήθηκα με ένα άλλο project της Eurostat, το ESDEN (European Statistical Data Exchange Network), το οποίο εξυπηρετεί 
κυρίως στην διαχείριση των web endpoints και των web server/client requests για την ανταλλαγή στατιστικών δεδομένων. Συγκεκριμένα, ασχολήθηκα 
με την προσαρμογή των test cases στην τρέχουσα version της εφαρμογής, καθώς κάποιες δεν εκτελούνταν σωστά. \\

Γενικότερα, στην διαδικασία του testing και ειδικά 
σε έναν agile τρόπο ανάπτυξης, σύμφωνα με τον οποίο η ομάδα κάνει μικρότερα και συχνά deliveries του προϊόντος, η εκτέλεση των API test cases είναι κάτι που 
εφαρμόζεται πολύ περισσότερο. Οι έλεγχοι της διεπαφής χρήστη απαιτούν περισσότερο χρόνο στην συγγραφή, την εκτέλεση και την συντήρηση, ενώ ελέγχουν την ίδια 
λειτουργικότητα με τα API test cases. Συνεπώς, τα UI test cases με τα οποία ασχολήθηκα εγώ είχαν αρκετό καιρό να εκτελεστούν οπότε χρειαζόντουσαν διορθώσεις.
Ασχολήθηκα με τροποποιήσεις σε elements της διεπαφής χρήστη και με την γενικότερη ροή της λειτουργικότητας του web app. Επικοινωνούσα, επίσης, με έναν συνάδελφο 
από την ομάδα μου όποτε χρειαζόμουν βοήθεια σε κάτι ή είχα κάποια απορία, και εκείνος ήταν πάντα πρόθυμος να με βοηθήσει και να μου εξηγήσει οτιδήποτε.
Η δραστηριότητα διήρκησε από τις 13/04/2022 έως τις 20/04/2022.

\section*{Τροποποίηση ενός Test Case Scenario βάσει νέων προδιαγραφών του πελάτη - ESDEN}
Όπως αναφέρθηκε παραπάνω, το ESDEN αποτελεί μια web εφαρμογή της Eurostat μέσω της οποία επιτυγχάνεται η διαχείριση και ανταλλαγή διάφορων στατιστικών datasets. Τα dataset 
αυτά μεταφέρονται μέσω HTTP requests μεταξύ web clients και web servers, τους οποίους διασυνδέει το ESDEN. Μετά από συζήτηση με τον πελάτη, η ομάδα μου ενημερώθηκε για μια 
τροποποίηση στον τρόπο με τον οποίο η εφαρμογή διαχειρίζεται τους διάφορους web servers για να κάνει κάποιες ενέργειες. Ένα συγκεκριμένο test case ελέγχει το σενάριο αποτυχημένης 
μεταφοράς κάποιου dataset μέσω ενός μη έγκυρου web server. Οι αλλαγές που ζητήθηκαν, λοιπόν, αφορούσαν τον τρόπο με τον οποίο η εφαρμογή δημιουργεί και χρησιμοποιεί 
τον server αυτόν. \\ 

Σε συνεργασία με έναν senior test automation engineer της ομάδας μου, δουλέψαμε πάνω στην ζητούμενη τροποποίηση. Τελικά διαπιστώθηκε πως αυτό που είχε ζητηθεί 
από τον πελάτη δεν ήταν απόλυτα υλοποιήσιμο ακριβώς με τον τρόπο που ζητούνταν. Συνεπώς, ο συνεργάτης μου θα μιλούσε και πάλι με τον πελάτη για να του μεταφέρει 
την πληροφορία και ,πιθανόν, να ακολουθηθεί στο προσεχές μέλλον μια διαφορετική προσέγγιση για την αντιμετώπιση του ζητήματος. Η δραστηριότητα διήρκησε από τις 19/04/2022 έως τις 20/04/2022.

\section*{Ενημέρωηση Test Case Status Report - SDMXRi}
Στη δραστηριότητα που ακολούθησε κλήθηκα να σχηματίσω μια αναφορά, η οποία περιείχε συγκεντρωμένα όλα τα test cases τόσο για την διεπαφή χρήστη 
όσο και για το ΑΡΙ της εφαρμογής SDMXRi. Συγκεκριμένα, μου ανατέθηκε να ενημερώσω το report αυτό με την κατάσταση των test cases καθώς, επίσης, και 
με κάποιες συμπληρωματικές πληροφορίες. Δηλαδή, εκεί περιέχονταν συνολικά όλα τα test cases με το εάν έτρεχαν επιτυχώς ή όχι. Σε περίπτωση 
προβλήματος ή αποτυχίας, εφόσον το issue είχε ήδη καταχωρηθεί, υπήρχε το αντίστοιχο ticket στο Jira (εργαλείο Project Management) όπου υπήρχαν περισσότερες λεπτομέρειες για το θέμα. \\

Για την σωστή συμπλήρωση όλου του report έπρεπε να εκτελεστούν όλα τα test cases της εφαρμογής (UI \& API), τα οποία ξεπερνούσαν τα 200 σε αριθμό και να σημειωθούν τυχόν 
προβλήματα και αδυναμίες στην εκτέλεση έτσι ώστε να προωθηθούν για διόρθωση. Το report αυτό βοήθησε στη συνέχεια την ομάδα μου στον καλύτερο προγραμματισμό και οργάνωση 
των επόμενων παραδοτέων που θα γίνονταν στον πελάτη, καθώς περιείχε συγκεντρωμένη την τρέχουσα κατάσταση της δουλειάς που είχε γίνει. Η δραστηριότητα διήρκησε από τι 21/04/2022 έως τις 26/04/2022.

\section*{Δημιουργία νέων UI Test Cases - ESDEN}
Κατά την δραστηριότητα αυτή ανέλαβα να δημιουργήσω από την αρχή πέντε καινούργιες περιπτώσεις ελέγχου για την εφαρμογή ESDEN. Αυτές οι test cases 
αφορούσαν την διεπαφή χρήστη της εφαρμογής. Οι προδιαγραφές των ελέγχων που έπρεπε να φτιάξω μου παραδώθηκαν μέσω του Jira στο οποίο είχαν καταχωρηθεί,  
και από όπου μπορούσα να δω ακριβώς τα βήματα που έπρεπε να εκτελετούν, τις προσυνθήκες, τις μετασυνθήκες κ.ά. όλα καθορισμένα σε φυσική γλώσσα. 
Κατά την συγγραφή του κώδικα, συμβουλευόμουν τον supervisor και συνεργάτη μου για τυχόν διευκρινήσεις τις οποίες μου παρείχε με μεγάλη προθυμία. 
Η δημιουργία των test cases διήρκησε από τις 26/04/2022 έως τις 03/05/2022. Στη συνέχεια, ο νέος κώδικας καταχωρήθηκε στο αποθετήριο της εφαρμογής 
στο Gitlab απ' όπου και θα γινόταν η διαδικασία του Code Review. Οι ημέρες από τις 04/05/2022 έως τις 06/05/2022 περιλάμβαναν προσαρμογές και τυχόν 
διορθώσεις στις καινούργιες test cases.

\section*{Συντήρηση των UI Test Cases - SDMXRi}
Αφού είχε πλέον δημιουργηθεί το Test Case Status Report που αναφέρθηκε παραπάνω, είχαν πλέον καταγραφεί κάποια νεά προβλήματα στην λειτουργικότητα 
των test cases. Στην δραστηριότητα αυτή, ανέλαβα να κάνω μια μικρή συντήρηση στα tets cases της διεπαφής χρήστη διορθώνοντας τα προβλήματα 
που εντοπίστηκαν κατά την δημιουργία του report. Ωστόσο, κάποιες από τις διορθώσεις, αποτελούσαν ζητήματα για τα οποία ευθύνονταν η λειτουργικότητα της 
εφαρμογής. Συνεπώς, αυτό σήμαινε πως έπρεπε να υπάρξει επικοινωνία με τους developers. Έτσι, διενεργήθηκε ένα meeting μεταξύ της ομάδας μου 
και ενός developer, με τον οποίο συζητήθηκαν και επιλύθηκαν τα θέματα που εντοπίστηκαν στην λειτορυγικότητα της εφαρμογής κατά την συντήρηση.
Η συντήρηση διενεργήθηκε στις 03/05/2022 και το meeting έγινε στις 06/05/2022.

\section*{API Testing Training and Troubleshooting - SDMXRi}
Στην δραστηριότητα αυτή ασχολήθηκα αποκλειστικά με το API testing. Μέχρι στιγμής δεν μου είχε δωθεί η ευκαιρία να δω με μεγάλη λεπτομέρεια 
τον τρόπο λειτουργίας των API tests. Έτσι, συνεργάστηκα με ένα από τα μέλη της ομάδας μου, το οποίο μέσω ορισμένων sessions μου έδειξε κάποια βασικά 
πράγματα για τη δομή του κώδικα, τη νοοτροπία των API tests και το πώς λειτουργούν. Μου έδειξε, επίσης, το πώς όλα τα παραπάνω χρησιμοποιούνται 
στην εφαρμογή SDMXRi. Μετά από το training, ασχολήθηκα με τη διόρθωση κάποιων bugs στα API tests που δεν έτρεχαν σωστά, σύμφωνα με το report 
που πλέον είχαμε στη διάθεσή μας. Σε συνεργασία, πάντα, με τον συνάδελφο συχνά χρειάστηκε να δουλέψουμε μαζί προσπαθώντας να βρούμε το πρόβλημα 
και το τι το προκαλεί. Η δραστηριότητα διήρκησε από τις 09/05/2022 έως τις 11/05/2022.

\section*{Διεξαγωγή εβδομαδιαίων development meetings - SDMXRi}
Στις 09/05/2022 ο Team Leader μου διοργάνωσε το πρώτο meeting που θα αφορά την πορεία όλων των projects της Eurostat. Στο meeting αυτό συμμετείχε η ομάδα 
μου που ασχολείται με τα συγκεκριμένα project καθώς επίσης και μια tester της εταιρείας από το εξωτερικό, η οποία πρόσφατα εντάχθηκε στα project της Eurostat. 
Το συγκεκριμένο meeting αποτέλεσε εν μέρει και ένα καλωσόρισμα στυο νέο μέλος της ομάδας και μας έδωσε την ευκαιρία να γνωριστούμε. Σκοπός του meeting είναι να 
οργανώνουμε την πορεία των εργασιών κάθε εβδομάδα, να ενημερώνουμε τους δυο leaders που συμμετέχουν για την πρόοδό τους καθώς επίσης και να συζητάμε τυχόν 
προβληματισμούς.