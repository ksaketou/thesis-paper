\setcounter{chapter}{0}
\chapter{Εισαγωγή}

Η εταιρεία στην οποία διενεργήθηκε η Πρακτική μου Άσκηση είναι η Netcompany-Intasoft. Η εταιρεία ιδρύθηκε το 1996 έχοντας την επωνυμία Intrasoft International με έδρα το Λουξεμβούργο και αποτελούσε μέλος του 
ομίλου εταιρειών Intracom Holdings. Απο τον Οκτώβριο του 2021, ωστόσο, έγινε μέλος της Netcompany Group, η οποία αποτελεί μια από τις πιο επιτυχημένες εταιρείες πληροφορικής του Βορρά με έδρα την Δανία με έτος 
ίδρυσης το 2000.\\

Η Netcompany-Intrasoft προσφέρει λύσεις και υπηρεσίες πληροφορικής σε 500 οργανισμούς 70 χωρών παγκοσμίως. Οι φορείς αυτοί δραστηριοποιούνται σε διάφορους κλάδους, όπως αυτόν του Δημόσιου Τομέα, της Ευρωπαϊκής 
Ένωσης, Τραπεζικής και Χρηματοιοκονομικών, Κοινωνικής και Υγειονομικής Ασφάλισης, Ενέργειας, Τηλεπικοινωνιών κ.ά. Η επιχείρηση απασχολεί περίπου 2800 εργαζόμενους 50 διαφορετικών εθνικοτήτων, το οποίο επιτυγχάνει 
διατηρώντας γραφεία σε 13 χώρες. Έτσι, εκδηλώνεται το ενδιαφέρον της εταιρίας για διατήρηση και υιοθέτηση της πολυπολιτισμικότητας και διαφορετικότητας στο περιβάλλον της, επιτυγχάνοντας με αυτόν τον τρόπο την 
ανταλλαγή διαφορετικών απόψεων και τη δημιουργική συνύπαρξη των ανθρώπων της σε ένα ενιαίο πλαίσιο.\\

Η εταιρεία με τοποθέτησε στο κομμάτι που παρέχει λύσεις στην Ευρωπαϊκή Επιτροπή (European Commission) και ,συγκεκριμένα, σε projects που αφορούν προϊόντα της Ευρωπαϊκής Στατιστικής Υπηρεσίας (Eurostat). Καθήκον 
της υπηρεσίας αυτής αποτελεί η συλλογή και δημοσίευση στατιστικών δεδομένων και μελετών που αφορούν τις χώρες της Ευρωπαϊκής Ένωσης. Οι web εφαρμογές του οργανισμού αυτού εξυπηρετούν στην διαχείριση και ανταλλαγή 
στατιστικών δεδομένων μεταξύ των χωρών.\\

Κατά την διάρκεια της πρακτικής άσκησης, απασχολήθηκα στο κομμάτι του Application Testing. Συγκεκριμένα, εντάχθηκα στο Software Testing Services Center (STSC), το οποίο αποτελεί ένα από τα μεγαλύτερα αυτόνομα 
Test Centers με πάνω από 100 Test Analysts και Test Engineers. Θέλοντας να ασχοληθώ με το κομμάτι του Test Development, απέκτησα τον ρόλο του Test Automation Engineer. Απασχολήθηκα, δηλαδή, σε οτιδήποτε αφορά 
έλεγχο λογισμικού μέσω αυτοματοποιημένων test scripts.\\

Στόχος της παρούσας τελικής αναφοράς είναι η συνολική παρουσίαση της συμμετοχής μου στο Πρόγραμμα Πρακτικής Άσκησης, μέσω της οποίας παρατίθενται οι δραστηριότητες και τα καθήκοντα που ανέλαβα κατά την συνεργασία 
μου με την εταιρία. Αναφέρονται, ακόμη, οι δεξιότητες και τα εργαλεία που αξιοποιήθηκαν το διάστημα αυτό, καθώς επίσης και οι γνώσεις που απέκτησα και ανέπτυξα μέσω της πρακτικής άσκησης. Τέλος, αναφέρεται και η 
εμπειρία μου από το εργασιακό περιβάλλον και τις συνεργασίες που ανέπτυξα κατά την παραμονή μου στην εταιρία.