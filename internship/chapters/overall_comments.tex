\chapter{Σχόλια - Παρατηρήσεις}

\section{Εργασιακό Περιβάλλον και Λειτουργία Τμήματος}
Στο συγκεκριμένο μέρος της αναφοράς παρουσιάζεται το γενικότερο περιβάλλον εργασίας μέσα στην εταιρεία αλλά και στο Software Testing 
Services Center. Ξεκινώντας από την πρώτη ημέρα έναρξης της πρακτικής άσκησης, οι εντυπώσεις μου από την υποδοχή και το καλωσόρισμα είναι 
αρκετά θετικές. Αφού συμπληρώθηκαν τα απαραίτητα έντυπα, μια συνάδελφος από τους team leaders με οδήγησε στα αντίστοιχα γραφεία του τμήματος. 
Όσο περίμενα για την παραλαβή του εξοπλισμού εργασίας μου, εκείνη με ξενάγησε στους χώρους του κτηρίου όπως επίσης και με σύστησε στους υπόλοιπους 
συναδέλφους που εργάζονταν στο γραφείο εκείνη την ημέρα. Ακόμη, με καλωσόρισε και η υπεύθυνη του τμήματος κα Χαρπίδη Φωτεινή. 
Μόλις παρέλαβα τον εξοπλισμό μου ασχολήθηκα με το set up του υπολογιστή, στο οποίο με βοήθησε 
με προθημία παράλληλα και το τμήμα του ΙΤ. Μόλις ολοκληρώθηκε το set up, επικοινώνησα με τον team leader των EuroStat projects, αλλά και με έναν 
ακόμη συνάδελφο της ομάδας μου, για να γνωριστούμε και να μου κάνουν μια αρχική εισαγωγή στις διαδικασίες του project. Ωστόσο, το υπόλοιπο της ημέρας ήταν αρκετά ήρεμο 
καθώς δεν μου είχε ανατεθεί ακόμα κάποια δραστηριότητα. Η εντύπωση που μου δημιουργήθηκε από την πρώτη ημέρα στην εταιρεία είναι πως οι εργασίες εκτελούνται χωρίς να επικρατεί 
πιεστικό κλίμα. Ακόμη, πέραν της επαγγελματικής συνεργασίας που είναι απαραίτητη μέσα στην εταιρία, υπάρχει ταυτόχρονα και φιλικό κλίμα μεταξύ των εργαζομένων.\\

Η παραπάνω εντύπωση διατηρήθηκε και τις υπόλοιπες ημέρες της πρακτικής άσκησης που ακολούθησαν. Τις πρώτες ημέρες εργασίας μου στο γραφείο που δεν είχα γνωριστεί ακόμη με τους συναδέλφους, 
υπήρξε έντονη διάθεση από εκείνους να γνωριστούμε και να ανταλλάξουμε απόψεις, όπως και έγινε. Κατά τη διάρκεια της πρακτικής άσκησης γνώρισα πολλά άτομα διαφορετικών ηλικιών, από διάφορες ομάδες αλλά και ρόλους, 
συζήτησα μαζί τους μαθαίνοντας, έτσι, νέα πράγματα για το αντικείμενό μου αλλά και τον εταιρικό χώρο. Συνεπώς, η διαδικασία αυτή αποτέλεσε σημαντική ευκαιρία για εμένα να διευρύνω τις γνώσεις μου και 
να εκτεθώ σε διαφορετικά ερεθίσματα μέσω νέων καταστάσεων.\\

Ταυτόχρονα, ιδιαίτερα ευχάριστο κλίμα επικρατεί και στις σχέσεις των εργαζομένων με τους team leaders αλλά και την manager του τμήματος. Κατά την αλληλεπίδραση και συνεργασία μου με τον leader της 
ομάδας στην οποία εντάχθηκα, δεν υπήρξε κάποια δυσάρεστη εμπειρία ενώ εκείνος ήταν πάντα πρόθυμος να με βοηθήσει με οποιαδήποτε απορία είχα, ιδιαίτερα τις πρώτες ημέρες που παράλληλα προσαρμοζόμουν 
στην δουλειά και τα νέα εργαλεία κατά την διάρκεια του training. Ακόμη, υπήρξε άμεση κατανόηση στα αιτήματα απουσίας μου από την δουλειά για τις ανάγκες των παρουσιάσεων προόδου στην σχολή.\\

Κατά την εκτέλεση των εργασιών που μου αναθέτονταν με βοήθησαν σημαντικά οι γνώσεις που έχω αποκτήσει από τις σπουδές μου στο ΔΕΤ. Διέθετα ήδη γνώση των βασικών εργαλείων που χρησιμοποιεί το STSC, δηλαδή 
Java, Git, βάσεις δεδομένων και αρχιτεκτονική web εφαρμογών. Τα εργαλεία αυτά αποτελούν παράλληλα την βάση για άλλα εργαλεία που χρησιμοποιεί το τμήμα όπως, για παράδειγμα, το Selenium. Μου ήταν, έτσι, πάρα πολύ 
εύκολο να μάθω τα υπόλοιπα εργαλεία κατά το training και να τα χρησιμοποιήσω αποτελεσματικά. Κατά τη διάρκεια της πρακτικής άσκησης, δεν αντιμετώπισα κάποια ιδιαίτερη δυσκολία σε επίπεδο γνώσεων. Αντιθέτως, πολλές 
φορές λάμβανα θετικό feedback από την ομάδα μου για την δυνατότητά μου να ανταπεξέρχομαι στις απαιτήσεις των projects. Ανάλογο feedback έλαβα, επίσης, και στο τέλος της πρακτικής άσκησης από τον team leader μου.\\

Παρ' όλα αυτά, δεν είχα κάποια προηγούμενη εμπειρία σε ότι αφορά τη γενικότερη διαχείριση εφαρμογών τόσο μεγάλης κλίμακας όπως της Eurostat. Έτσι, δεν είχα τις αντίστοιχες γνώσεις σχετικά με όλη τη 
διαδικασία και τις πρακτικές που ακολουθούνται για το deployment ενος νέου release, τις καλές και κακές πρακτικές σε ότι αφορά συντήριση του κώδικα αλλά και της βάσης κ.ά. Ωστόσο, αντίστοιχες γνώσεις 
απέκτησα σταδιακά κατά την διάρκεια της δουλειάς μέσω συνεργασίας, αποριών και συζητήσεων με τα μέλη της ομάδας μου.

\section{Δεξιότητες}
Κατά τη διάρκεια της πρακτικής άσκησης, απέκτησα τεχνικές δεξιότητες μέσω των εργασιών που εκτελούσα καθημερινά. Ακόμη, η γνωριμία, συνεργασία και συνύπαρξη με άτομα του αντίστοιχου κλάδου ανέπτυξε και τα 
soft skills μου με διάφορους τρόπους. Ξεκινώντας από τις τεχνικές γνώσεις, παρακάτω παρατίθενται εργαλεία τα οποία ήδη γνώριζα και εξέλιξα ή έμαθα μέσα από την δουλειά.
\begin{itemize}
    \item Java: Η γλώσσα προγραμματισμού που χρησιμοποίησα κυρίως στις δραστηριότητες που ανέλαβα
    \item Git/Gitlab: Το εργαλείο που χρησιμοποιείται για Version Control του κώδικα
    \item Βάσεις Δεδομένων (mySQL, SQLServer, ORACLE DB): Απαραίτητο εργαλείο για την αποθήκευση των test data των εφαρμογών
    \item Selenium: Απαραίτητο εργαλείο για test automation tasks
    \item API Testing: Εξοικείωση με μεθοδολογίες, εργαλεία και γενικότερες γνώσεις στον τρόπο λειτουργίας και τη δομή των APIs, των web requests κ.ά.
    \item Αρχικτεκτονική Web Εφαρμογών: Απαραίτητη γνώση για την κατανόηση του τρόπου λειτουργίας των εφαρμογών που πρόκειται να ελεγχθούν
\end{itemize}

Όσον αφορά τα soft skills, μου δώθηκε η ευκαιρία να εξελίξω εκείνα που ήδη διέθετα όπως η εργασία ως μέρος μιας ομάδας, η ικανότητα αποτελεσματικής επικοινωνίας με τους συναδέλφους και 
η αποδοτική οργάνωση των εργασιών σύμφωνα με τον διαθέσιμο χρόνο. Σημαντικό ρόλο στις παραπάνω δεξιότητες έπαιξαν, επίσης, και τα εβδομαδιαία meetings που είχαμε με τους team leaders όπου έπρεπε να παρουσιαστεί 
συνοπτικά η πρόοδος όλης της εβδομάδας αλλά και τα tasks που ακολουθούσαν. Ακόμη, κατά τη διάρκεια της πρακτικής, έλαβα ιδιαίτερη εμπιστοσύνη από τους συναδέλφους μου σε ότι αφορά την ανάθεση και συμμετοχή μου σε 
σημαντικές δραστηριότητες όπως το refactoring ενός testing framework, γεγονός το οποίο εκτίμησα ιδιαιτέρως. Χάρη σε αυτό, ανέπτυξα την κριτική μου σκέψη καθώς πολλές φορές κλήθηκα να λάβω αποφάσεις  
όσον αφορά την εφαρμογή αλλαγών στον κώδικα και την επικοινωνία αντίστοιχων προτάσεων για βελτίωση στα μέλη της ομάδας μου.\\

Μετά την ολοκλήρωση της πρακτικής άσκησης, συνέχισα την συνεργασία μου με την εταιρία στη θέση του Junior SW Automation Engineer.
